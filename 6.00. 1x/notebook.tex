
% Default to the notebook output style

    


% Inherit from the specified cell style.




    
\documentclass[11pt]{article}

    
    
    \usepackage[T1]{fontenc}
    % Nicer default font (+ math font) than Computer Modern for most use cases
    \usepackage{mathpazo}

    % Basic figure setup, for now with no caption control since it's done
    % automatically by Pandoc (which extracts ![](path) syntax from Markdown).
    \usepackage{graphicx}
    % We will generate all images so they have a width \maxwidth. This means
    % that they will get their normal width if they fit onto the page, but
    % are scaled down if they would overflow the margins.
    \makeatletter
    \def\maxwidth{\ifdim\Gin@nat@width>\linewidth\linewidth
    \else\Gin@nat@width\fi}
    \makeatother
    \let\Oldincludegraphics\includegraphics
    % Set max figure width to be 80% of text width, for now hardcoded.
    \renewcommand{\includegraphics}[1]{\Oldincludegraphics[width=.8\maxwidth]{#1}}
    % Ensure that by default, figures have no caption (until we provide a
    % proper Figure object with a Caption API and a way to capture that
    % in the conversion process - todo).
    \usepackage{caption}
    \DeclareCaptionLabelFormat{nolabel}{}
    \captionsetup{labelformat=nolabel}

    \usepackage{adjustbox} % Used to constrain images to a maximum size 
    \usepackage{xcolor} % Allow colors to be defined
    \usepackage{enumerate} % Needed for markdown enumerations to work
    \usepackage{geometry} % Used to adjust the document margins
    \usepackage{amsmath} % Equations
    \usepackage{amssymb} % Equations
    \usepackage{textcomp} % defines textquotesingle
    % Hack from http://tex.stackexchange.com/a/47451/13684:
    \AtBeginDocument{%
        \def\PYZsq{\textquotesingle}% Upright quotes in Pygmentized code
    }
    \usepackage{upquote} % Upright quotes for verbatim code
    \usepackage{eurosym} % defines \euro
    \usepackage[mathletters]{ucs} % Extended unicode (utf-8) support
    \usepackage[utf8x]{inputenc} % Allow utf-8 characters in the tex document
    \usepackage{fancyvrb} % verbatim replacement that allows latex
    \usepackage{grffile} % extends the file name processing of package graphics 
                         % to support a larger range 
    % The hyperref package gives us a pdf with properly built
    % internal navigation ('pdf bookmarks' for the table of contents,
    % internal cross-reference links, web links for URLs, etc.)
    \usepackage{hyperref}
    \usepackage{longtable} % longtable support required by pandoc >1.10
    \usepackage{booktabs}  % table support for pandoc > 1.12.2
    \usepackage[inline]{enumitem} % IRkernel/repr support (it uses the enumerate* environment)
    \usepackage[normalem]{ulem} % ulem is needed to support strikethroughs (\sout)
                                % normalem makes italics be italics, not underlines
    

    
    
    % Colors for the hyperref package
    \definecolor{urlcolor}{rgb}{0,.145,.698}
    \definecolor{linkcolor}{rgb}{.71,0.21,0.01}
    \definecolor{citecolor}{rgb}{.12,.54,.11}

    % ANSI colors
    \definecolor{ansi-black}{HTML}{3E424D}
    \definecolor{ansi-black-intense}{HTML}{282C36}
    \definecolor{ansi-red}{HTML}{E75C58}
    \definecolor{ansi-red-intense}{HTML}{B22B31}
    \definecolor{ansi-green}{HTML}{00A250}
    \definecolor{ansi-green-intense}{HTML}{007427}
    \definecolor{ansi-yellow}{HTML}{DDB62B}
    \definecolor{ansi-yellow-intense}{HTML}{B27D12}
    \definecolor{ansi-blue}{HTML}{208FFB}
    \definecolor{ansi-blue-intense}{HTML}{0065CA}
    \definecolor{ansi-magenta}{HTML}{D160C4}
    \definecolor{ansi-magenta-intense}{HTML}{A03196}
    \definecolor{ansi-cyan}{HTML}{60C6C8}
    \definecolor{ansi-cyan-intense}{HTML}{258F8F}
    \definecolor{ansi-white}{HTML}{C5C1B4}
    \definecolor{ansi-white-intense}{HTML}{A1A6B2}

    % commands and environments needed by pandoc snippets
    % extracted from the output of `pandoc -s`
    \providecommand{\tightlist}{%
      \setlength{\itemsep}{0pt}\setlength{\parskip}{0pt}}
    \DefineVerbatimEnvironment{Highlighting}{Verbatim}{commandchars=\\\{\}}
    % Add ',fontsize=\small' for more characters per line
    \newenvironment{Shaded}{}{}
    \newcommand{\KeywordTok}[1]{\textcolor[rgb]{0.00,0.44,0.13}{\textbf{{#1}}}}
    \newcommand{\DataTypeTok}[1]{\textcolor[rgb]{0.56,0.13,0.00}{{#1}}}
    \newcommand{\DecValTok}[1]{\textcolor[rgb]{0.25,0.63,0.44}{{#1}}}
    \newcommand{\BaseNTok}[1]{\textcolor[rgb]{0.25,0.63,0.44}{{#1}}}
    \newcommand{\FloatTok}[1]{\textcolor[rgb]{0.25,0.63,0.44}{{#1}}}
    \newcommand{\CharTok}[1]{\textcolor[rgb]{0.25,0.44,0.63}{{#1}}}
    \newcommand{\StringTok}[1]{\textcolor[rgb]{0.25,0.44,0.63}{{#1}}}
    \newcommand{\CommentTok}[1]{\textcolor[rgb]{0.38,0.63,0.69}{\textit{{#1}}}}
    \newcommand{\OtherTok}[1]{\textcolor[rgb]{0.00,0.44,0.13}{{#1}}}
    \newcommand{\AlertTok}[1]{\textcolor[rgb]{1.00,0.00,0.00}{\textbf{{#1}}}}
    \newcommand{\FunctionTok}[1]{\textcolor[rgb]{0.02,0.16,0.49}{{#1}}}
    \newcommand{\RegionMarkerTok}[1]{{#1}}
    \newcommand{\ErrorTok}[1]{\textcolor[rgb]{1.00,0.00,0.00}{\textbf{{#1}}}}
    \newcommand{\NormalTok}[1]{{#1}}
    
    % Additional commands for more recent versions of Pandoc
    \newcommand{\ConstantTok}[1]{\textcolor[rgb]{0.53,0.00,0.00}{{#1}}}
    \newcommand{\SpecialCharTok}[1]{\textcolor[rgb]{0.25,0.44,0.63}{{#1}}}
    \newcommand{\VerbatimStringTok}[1]{\textcolor[rgb]{0.25,0.44,0.63}{{#1}}}
    \newcommand{\SpecialStringTok}[1]{\textcolor[rgb]{0.73,0.40,0.53}{{#1}}}
    \newcommand{\ImportTok}[1]{{#1}}
    \newcommand{\DocumentationTok}[1]{\textcolor[rgb]{0.73,0.13,0.13}{\textit{{#1}}}}
    \newcommand{\AnnotationTok}[1]{\textcolor[rgb]{0.38,0.63,0.69}{\textbf{\textit{{#1}}}}}
    \newcommand{\CommentVarTok}[1]{\textcolor[rgb]{0.38,0.63,0.69}{\textbf{\textit{{#1}}}}}
    \newcommand{\VariableTok}[1]{\textcolor[rgb]{0.10,0.09,0.49}{{#1}}}
    \newcommand{\ControlFlowTok}[1]{\textcolor[rgb]{0.00,0.44,0.13}{\textbf{{#1}}}}
    \newcommand{\OperatorTok}[1]{\textcolor[rgb]{0.40,0.40,0.40}{{#1}}}
    \newcommand{\BuiltInTok}[1]{{#1}}
    \newcommand{\ExtensionTok}[1]{{#1}}
    \newcommand{\PreprocessorTok}[1]{\textcolor[rgb]{0.74,0.48,0.00}{{#1}}}
    \newcommand{\AttributeTok}[1]{\textcolor[rgb]{0.49,0.56,0.16}{{#1}}}
    \newcommand{\InformationTok}[1]{\textcolor[rgb]{0.38,0.63,0.69}{\textbf{\textit{{#1}}}}}
    \newcommand{\WarningTok}[1]{\textcolor[rgb]{0.38,0.63,0.69}{\textbf{\textit{{#1}}}}}
    
    
    % Define a nice break command that doesn't care if a line doesn't already
    % exist.
    \def\br{\hspace*{\fill} \\* }
    % Math Jax compatability definitions
    \def\gt{>}
    \def\lt{<}
    % Document parameters
    \title{Week 1}
    
    
    

    % Pygments definitions
    
\makeatletter
\def\PY@reset{\let\PY@it=\relax \let\PY@bf=\relax%
    \let\PY@ul=\relax \let\PY@tc=\relax%
    \let\PY@bc=\relax \let\PY@ff=\relax}
\def\PY@tok#1{\csname PY@tok@#1\endcsname}
\def\PY@toks#1+{\ifx\relax#1\empty\else%
    \PY@tok{#1}\expandafter\PY@toks\fi}
\def\PY@do#1{\PY@bc{\PY@tc{\PY@ul{%
    \PY@it{\PY@bf{\PY@ff{#1}}}}}}}
\def\PY#1#2{\PY@reset\PY@toks#1+\relax+\PY@do{#2}}

\expandafter\def\csname PY@tok@w\endcsname{\def\PY@tc##1{\textcolor[rgb]{0.73,0.73,0.73}{##1}}}
\expandafter\def\csname PY@tok@c\endcsname{\let\PY@it=\textit\def\PY@tc##1{\textcolor[rgb]{0.25,0.50,0.50}{##1}}}
\expandafter\def\csname PY@tok@cp\endcsname{\def\PY@tc##1{\textcolor[rgb]{0.74,0.48,0.00}{##1}}}
\expandafter\def\csname PY@tok@k\endcsname{\let\PY@bf=\textbf\def\PY@tc##1{\textcolor[rgb]{0.00,0.50,0.00}{##1}}}
\expandafter\def\csname PY@tok@kp\endcsname{\def\PY@tc##1{\textcolor[rgb]{0.00,0.50,0.00}{##1}}}
\expandafter\def\csname PY@tok@kt\endcsname{\def\PY@tc##1{\textcolor[rgb]{0.69,0.00,0.25}{##1}}}
\expandafter\def\csname PY@tok@o\endcsname{\def\PY@tc##1{\textcolor[rgb]{0.40,0.40,0.40}{##1}}}
\expandafter\def\csname PY@tok@ow\endcsname{\let\PY@bf=\textbf\def\PY@tc##1{\textcolor[rgb]{0.67,0.13,1.00}{##1}}}
\expandafter\def\csname PY@tok@nb\endcsname{\def\PY@tc##1{\textcolor[rgb]{0.00,0.50,0.00}{##1}}}
\expandafter\def\csname PY@tok@nf\endcsname{\def\PY@tc##1{\textcolor[rgb]{0.00,0.00,1.00}{##1}}}
\expandafter\def\csname PY@tok@nc\endcsname{\let\PY@bf=\textbf\def\PY@tc##1{\textcolor[rgb]{0.00,0.00,1.00}{##1}}}
\expandafter\def\csname PY@tok@nn\endcsname{\let\PY@bf=\textbf\def\PY@tc##1{\textcolor[rgb]{0.00,0.00,1.00}{##1}}}
\expandafter\def\csname PY@tok@ne\endcsname{\let\PY@bf=\textbf\def\PY@tc##1{\textcolor[rgb]{0.82,0.25,0.23}{##1}}}
\expandafter\def\csname PY@tok@nv\endcsname{\def\PY@tc##1{\textcolor[rgb]{0.10,0.09,0.49}{##1}}}
\expandafter\def\csname PY@tok@no\endcsname{\def\PY@tc##1{\textcolor[rgb]{0.53,0.00,0.00}{##1}}}
\expandafter\def\csname PY@tok@nl\endcsname{\def\PY@tc##1{\textcolor[rgb]{0.63,0.63,0.00}{##1}}}
\expandafter\def\csname PY@tok@ni\endcsname{\let\PY@bf=\textbf\def\PY@tc##1{\textcolor[rgb]{0.60,0.60,0.60}{##1}}}
\expandafter\def\csname PY@tok@na\endcsname{\def\PY@tc##1{\textcolor[rgb]{0.49,0.56,0.16}{##1}}}
\expandafter\def\csname PY@tok@nt\endcsname{\let\PY@bf=\textbf\def\PY@tc##1{\textcolor[rgb]{0.00,0.50,0.00}{##1}}}
\expandafter\def\csname PY@tok@nd\endcsname{\def\PY@tc##1{\textcolor[rgb]{0.67,0.13,1.00}{##1}}}
\expandafter\def\csname PY@tok@s\endcsname{\def\PY@tc##1{\textcolor[rgb]{0.73,0.13,0.13}{##1}}}
\expandafter\def\csname PY@tok@sd\endcsname{\let\PY@it=\textit\def\PY@tc##1{\textcolor[rgb]{0.73,0.13,0.13}{##1}}}
\expandafter\def\csname PY@tok@si\endcsname{\let\PY@bf=\textbf\def\PY@tc##1{\textcolor[rgb]{0.73,0.40,0.53}{##1}}}
\expandafter\def\csname PY@tok@se\endcsname{\let\PY@bf=\textbf\def\PY@tc##1{\textcolor[rgb]{0.73,0.40,0.13}{##1}}}
\expandafter\def\csname PY@tok@sr\endcsname{\def\PY@tc##1{\textcolor[rgb]{0.73,0.40,0.53}{##1}}}
\expandafter\def\csname PY@tok@ss\endcsname{\def\PY@tc##1{\textcolor[rgb]{0.10,0.09,0.49}{##1}}}
\expandafter\def\csname PY@tok@sx\endcsname{\def\PY@tc##1{\textcolor[rgb]{0.00,0.50,0.00}{##1}}}
\expandafter\def\csname PY@tok@m\endcsname{\def\PY@tc##1{\textcolor[rgb]{0.40,0.40,0.40}{##1}}}
\expandafter\def\csname PY@tok@gh\endcsname{\let\PY@bf=\textbf\def\PY@tc##1{\textcolor[rgb]{0.00,0.00,0.50}{##1}}}
\expandafter\def\csname PY@tok@gu\endcsname{\let\PY@bf=\textbf\def\PY@tc##1{\textcolor[rgb]{0.50,0.00,0.50}{##1}}}
\expandafter\def\csname PY@tok@gd\endcsname{\def\PY@tc##1{\textcolor[rgb]{0.63,0.00,0.00}{##1}}}
\expandafter\def\csname PY@tok@gi\endcsname{\def\PY@tc##1{\textcolor[rgb]{0.00,0.63,0.00}{##1}}}
\expandafter\def\csname PY@tok@gr\endcsname{\def\PY@tc##1{\textcolor[rgb]{1.00,0.00,0.00}{##1}}}
\expandafter\def\csname PY@tok@ge\endcsname{\let\PY@it=\textit}
\expandafter\def\csname PY@tok@gs\endcsname{\let\PY@bf=\textbf}
\expandafter\def\csname PY@tok@gp\endcsname{\let\PY@bf=\textbf\def\PY@tc##1{\textcolor[rgb]{0.00,0.00,0.50}{##1}}}
\expandafter\def\csname PY@tok@go\endcsname{\def\PY@tc##1{\textcolor[rgb]{0.53,0.53,0.53}{##1}}}
\expandafter\def\csname PY@tok@gt\endcsname{\def\PY@tc##1{\textcolor[rgb]{0.00,0.27,0.87}{##1}}}
\expandafter\def\csname PY@tok@err\endcsname{\def\PY@bc##1{\setlength{\fboxsep}{0pt}\fcolorbox[rgb]{1.00,0.00,0.00}{1,1,1}{\strut ##1}}}
\expandafter\def\csname PY@tok@kc\endcsname{\let\PY@bf=\textbf\def\PY@tc##1{\textcolor[rgb]{0.00,0.50,0.00}{##1}}}
\expandafter\def\csname PY@tok@kd\endcsname{\let\PY@bf=\textbf\def\PY@tc##1{\textcolor[rgb]{0.00,0.50,0.00}{##1}}}
\expandafter\def\csname PY@tok@kn\endcsname{\let\PY@bf=\textbf\def\PY@tc##1{\textcolor[rgb]{0.00,0.50,0.00}{##1}}}
\expandafter\def\csname PY@tok@kr\endcsname{\let\PY@bf=\textbf\def\PY@tc##1{\textcolor[rgb]{0.00,0.50,0.00}{##1}}}
\expandafter\def\csname PY@tok@bp\endcsname{\def\PY@tc##1{\textcolor[rgb]{0.00,0.50,0.00}{##1}}}
\expandafter\def\csname PY@tok@fm\endcsname{\def\PY@tc##1{\textcolor[rgb]{0.00,0.00,1.00}{##1}}}
\expandafter\def\csname PY@tok@vc\endcsname{\def\PY@tc##1{\textcolor[rgb]{0.10,0.09,0.49}{##1}}}
\expandafter\def\csname PY@tok@vg\endcsname{\def\PY@tc##1{\textcolor[rgb]{0.10,0.09,0.49}{##1}}}
\expandafter\def\csname PY@tok@vi\endcsname{\def\PY@tc##1{\textcolor[rgb]{0.10,0.09,0.49}{##1}}}
\expandafter\def\csname PY@tok@vm\endcsname{\def\PY@tc##1{\textcolor[rgb]{0.10,0.09,0.49}{##1}}}
\expandafter\def\csname PY@tok@sa\endcsname{\def\PY@tc##1{\textcolor[rgb]{0.73,0.13,0.13}{##1}}}
\expandafter\def\csname PY@tok@sb\endcsname{\def\PY@tc##1{\textcolor[rgb]{0.73,0.13,0.13}{##1}}}
\expandafter\def\csname PY@tok@sc\endcsname{\def\PY@tc##1{\textcolor[rgb]{0.73,0.13,0.13}{##1}}}
\expandafter\def\csname PY@tok@dl\endcsname{\def\PY@tc##1{\textcolor[rgb]{0.73,0.13,0.13}{##1}}}
\expandafter\def\csname PY@tok@s2\endcsname{\def\PY@tc##1{\textcolor[rgb]{0.73,0.13,0.13}{##1}}}
\expandafter\def\csname PY@tok@sh\endcsname{\def\PY@tc##1{\textcolor[rgb]{0.73,0.13,0.13}{##1}}}
\expandafter\def\csname PY@tok@s1\endcsname{\def\PY@tc##1{\textcolor[rgb]{0.73,0.13,0.13}{##1}}}
\expandafter\def\csname PY@tok@mb\endcsname{\def\PY@tc##1{\textcolor[rgb]{0.40,0.40,0.40}{##1}}}
\expandafter\def\csname PY@tok@mf\endcsname{\def\PY@tc##1{\textcolor[rgb]{0.40,0.40,0.40}{##1}}}
\expandafter\def\csname PY@tok@mh\endcsname{\def\PY@tc##1{\textcolor[rgb]{0.40,0.40,0.40}{##1}}}
\expandafter\def\csname PY@tok@mi\endcsname{\def\PY@tc##1{\textcolor[rgb]{0.40,0.40,0.40}{##1}}}
\expandafter\def\csname PY@tok@il\endcsname{\def\PY@tc##1{\textcolor[rgb]{0.40,0.40,0.40}{##1}}}
\expandafter\def\csname PY@tok@mo\endcsname{\def\PY@tc##1{\textcolor[rgb]{0.40,0.40,0.40}{##1}}}
\expandafter\def\csname PY@tok@ch\endcsname{\let\PY@it=\textit\def\PY@tc##1{\textcolor[rgb]{0.25,0.50,0.50}{##1}}}
\expandafter\def\csname PY@tok@cm\endcsname{\let\PY@it=\textit\def\PY@tc##1{\textcolor[rgb]{0.25,0.50,0.50}{##1}}}
\expandafter\def\csname PY@tok@cpf\endcsname{\let\PY@it=\textit\def\PY@tc##1{\textcolor[rgb]{0.25,0.50,0.50}{##1}}}
\expandafter\def\csname PY@tok@c1\endcsname{\let\PY@it=\textit\def\PY@tc##1{\textcolor[rgb]{0.25,0.50,0.50}{##1}}}
\expandafter\def\csname PY@tok@cs\endcsname{\let\PY@it=\textit\def\PY@tc##1{\textcolor[rgb]{0.25,0.50,0.50}{##1}}}

\def\PYZbs{\char`\\}
\def\PYZus{\char`\_}
\def\PYZob{\char`\{}
\def\PYZcb{\char`\}}
\def\PYZca{\char`\^}
\def\PYZam{\char`\&}
\def\PYZlt{\char`\<}
\def\PYZgt{\char`\>}
\def\PYZsh{\char`\#}
\def\PYZpc{\char`\%}
\def\PYZdl{\char`\$}
\def\PYZhy{\char`\-}
\def\PYZsq{\char`\'}
\def\PYZdq{\char`\"}
\def\PYZti{\char`\~}
% for compatibility with earlier versions
\def\PYZat{@}
\def\PYZlb{[}
\def\PYZrb{]}
\makeatother


    % Exact colors from NB
    \definecolor{incolor}{rgb}{0.0, 0.0, 0.5}
    \definecolor{outcolor}{rgb}{0.545, 0.0, 0.0}



    
    % Prevent overflowing lines due to hard-to-break entities
    \sloppy 
    % Setup hyperref package
    \hypersetup{
      breaklinks=true,  % so long urls are correctly broken across lines
      colorlinks=true,
      urlcolor=urlcolor,
      linkcolor=linkcolor,
      citecolor=citecolor,
      }
    % Slightly bigger margins than the latex defaults
    
    \geometry{verbose,tmargin=1in,bmargin=1in,lmargin=1in,rmargin=1in}
    
    

    \begin{document}
    
    
    \maketitle
    
    

    
    \hypertarget{book-notes}{%
\section{Book Notes}\label{book-notes}}

    \hypertarget{quad-some-simple-numerical-programs}{%
\section{\texorpdfstring{3\(\quad\) SOME SIMPLE NUMERICAL
PROGRAMS}{3\textbackslash{}quad SOME SIMPLE NUMERICAL PROGRAMS}}\label{quad-some-simple-numerical-programs}}

\begin{center}\rule{0.5\linewidth}{\linethickness}\end{center}

\hypertarget{quad-exhaustive-enumeration}{%
\subsection{\texorpdfstring{3.1\(\quad\) Exhaustive
Enumeration}{3.1\textbackslash{}quad Exhaustive Enumeration}}\label{quad-exhaustive-enumeration}}

The code below finds the cube root of a perfect cube, \textbf{using
exhaustive enumeration to find the cube root}

    \begin{Verbatim}[commandchars=\\\{\}]
{\color{incolor}In [{\color{incolor}5}]:} \PY{c+c1}{\PYZsh{}Find the cube root of a perfect cube}
        \PY{n}{x} \PY{o}{=} \PY{n+nb}{int}\PY{p}{(}\PY{n+nb}{input}\PY{p}{(}\PY{l+s+s1}{\PYZsq{}}\PY{l+s+s1}{Enter an integer: }\PY{l+s+s1}{\PYZsq{}}\PY{p}{)}\PY{p}{)}
        \PY{n}{ans} \PY{o}{=} \PY{l+m+mi}{0}
        \PY{k}{while} \PY{n}{ans}\PY{o}{*}\PY{o}{*}\PY{l+m+mi}{3} \PY{o}{\PYZlt{}} \PY{n+nb}{abs}\PY{p}{(}\PY{n}{x}\PY{p}{)}\PY{p}{:}
            \PY{n}{ans} \PY{o}{+}\PY{o}{=} \PY{l+m+mi}{1}
        \PY{k}{if} \PY{n}{ans}\PY{o}{*}\PY{o}{*}\PY{l+m+mi}{3} \PY{o}{!=} \PY{n+nb}{abs}\PY{p}{(}\PY{n}{x}\PY{p}{)}\PY{p}{:}
            \PY{n+nb}{print}\PY{p}{(}\PY{n}{x}\PY{p}{,} \PY{l+s+s1}{\PYZsq{}}\PY{l+s+s1}{is not a perfect cube}\PY{l+s+s1}{\PYZsq{}}\PY{p}{)}
        \PY{k}{else}\PY{p}{:}
            \PY{k}{if} \PY{n}{x} \PY{o}{\PYZlt{}} \PY{l+m+mi}{0}\PY{p}{:}
                \PY{n}{ans} \PY{o}{=} \PY{o}{\PYZhy{}}\PY{n}{ans}
            \PY{n+nb}{print}\PY{p}{(}\PY{l+s+s1}{\PYZsq{}}\PY{l+s+s1}{Cube root of}\PY{l+s+s1}{\PYZsq{}}\PY{p}{,} \PY{n}{x}\PY{p}{,} \PY{l+s+s1}{\PYZsq{}}\PY{l+s+s1}{is}\PY{l+s+s1}{\PYZsq{}}\PY{p}{,} \PY{n}{ans}\PY{p}{)}
\end{Verbatim}


    \begin{Verbatim}[commandchars=\\\{\}]
Enter an integer: 343
Cube root of 343 is 7

    \end{Verbatim}

    For what values of \(x\) will this program terminate?

The answer is, ``all integers.'' This can be argued quite simply. * The
value of the expression ans**3 starts at 0, and gets larger each time
through the loop. * When it reaches or exceeds abs(x), the loop
terminates. * Since abs(x) is always positive there are only a finite
number of iterations before the loop must terminate.

Whenever you write a loop, you should think about an appropriate
\textbf{decrementing function}. This is a function that has the
following properties: 1. It maps a set of program variables into an
integer. 2. When the loop is entered, its value is nonnegative. 3. When
its value is \textless{}=0, the loop terminates. 4. Its value is
decreased everytime through the loop.

What is the decrementing function for the loop above? It is
\texttt{abs(x)\ -\ ans**3}.

If you add some errors, this is what happens. If you comment out
\texttt{ans\ =\ 0}, then the interpreter gives an error because
\texttt{ans} is not defined. If you replace \texttt{ans\ =\ ans\ +\ 1}
with \texttt{ans\ =\ ans}, you're stuck in an infinite loop. Ctrl + c
will get you out of the infinite loop in like, iPython and/or Spyder,
but I guess not in jupyter notebook.

If you add
\texttt{print(\textquotesingle{}Value\ of\ the\ decrementing\ function\ abs(x)\ -\ ans**3\ is\textquotesingle{},\textbackslash{}abs(x)\ -\ ans**3)}
at the start of the loop, then it'll print that over and over again. The
program would have run forever because the loop body is no longer
reducing the distance between \texttt{ans**3} and \texttt{abs(x)}.

\textbf{When confronted with a program that seems not to be terminating,
experienced programmers often insert print statements, such as the one
here, to test whether the decrementing function is indeed being
decremented.}

The algorithmic technique used in this program is a variant of
\textbf{guess and check} called \textbf{exhaustive enumeration}. We
enumerate all possibilities until we get to the right answer of exhaust
the space of possibilities.

    \begin{Shaded}
\begin{Highlighting}[]
\CommentTok{#This tip on inserting print statements to debug my code which wasn't terminating was very useful, helped me out in the below finger exercise.}
\end{Highlighting}
\end{Shaded}

    \begin{Verbatim}[commandchars=\\\{\}]
{\color{incolor}In [{\color{incolor}6}]:} \PY{n+nb}{max} \PY{o}{=} \PY{n+nb}{int}\PY{p}{(}\PY{n+nb}{input}\PY{p}{(}\PY{l+s+s1}{\PYZsq{}}\PY{l+s+s1}{Enter a positive integer: }\PY{l+s+s1}{\PYZsq{}}\PY{p}{)}\PY{p}{)}
        \PY{n}{i} \PY{o}{=} \PY{l+m+mi}{0}
        \PY{k}{while} \PY{n}{i} \PY{o}{\PYZlt{}} \PY{n+nb}{max}\PY{p}{:}
            \PY{n}{i} \PY{o}{+}\PY{o}{=} \PY{l+m+mi}{1}
        \PY{n+nb}{print}\PY{p}{(}\PY{n}{i}\PY{p}{)}
\end{Verbatim}


    \begin{Verbatim}[commandchars=\\\{\}]
Enter a positive integer: 10000
10000

    \end{Verbatim}

    See how large an integer you need to enter before there's a perceptible
pause before the result is printed.\\
\emph{There was a perceptible pause at around 10000000.}

    \begin{Shaded}
\begin{Highlighting}[]
\CommentTok{#Indentation is important. if the last line, `print(i)` is on the same indentation as the previous line, the program will print 1 2 3 4 5, if `print(i)` is outside, far left, like the book shows, then it only prints 5, given input = 5.}
\end{Highlighting}
\end{Shaded}

    \textbf{Finger exercise:} Write a program that asks the user to enter an
integer and prints two integers, \texttt{root} and \texttt{pwr} such
that \texttt{0\ \textless{}\ pwr\ \textless{}\ 6} and \texttt{root**pwr}
is equal to the integer entered by the user. If no such pair of integers
exists, it should print a message to that effect.

    \begin{Verbatim}[commandchars=\\\{\}]
{\color{incolor}In [{\color{incolor}7}]:} \PY{n}{integer} \PY{o}{=} \PY{n+nb}{int}\PY{p}{(}\PY{n+nb}{input}\PY{p}{(}\PY{l+s+s1}{\PYZsq{}}\PY{l+s+s1}{Enter an integer: }\PY{l+s+s1}{\PYZsq{}}\PY{p}{)}\PY{p}{)}
        \PY{n}{pwr} \PY{o}{=} \PY{l+m+mi}{0}
        \PY{n}{result} \PY{o}{=} \PY{k+kc}{False}
        \PY{k}{while} \PY{n}{pwr} \PY{o}{\PYZlt{}} \PY{l+m+mi}{6}\PY{p}{:}
            \PY{n}{root} \PY{o}{=} \PY{n}{integer}
            \PY{k}{while} \PY{n}{root} \PY{o}{\PYZgt{}} \PY{l+m+mi}{0}\PY{p}{:}
                \PY{k}{if} \PY{n}{root}\PY{o}{*}\PY{o}{*}\PY{n}{pwr} \PY{o}{==} \PY{n}{integer}\PY{p}{:}
                    \PY{n}{result} \PY{o}{=} \PY{k+kc}{True}
                    \PY{n+nb}{print}\PY{p}{(}\PY{l+s+s2}{\PYZdq{}}\PY{l+s+s2}{root:}\PY{l+s+s2}{\PYZdq{}}\PY{p}{,} \PY{n}{root}\PY{p}{,} \PY{l+s+s2}{\PYZdq{}}\PY{l+s+s2}{power:}\PY{l+s+s2}{\PYZdq{}}\PY{p}{,} \PY{n}{pwr}\PY{p}{)}
                    \PY{n}{root} \PY{o}{\PYZhy{}}\PY{o}{=} \PY{l+m+mi}{1}
                \PY{k}{else}\PY{p}{:}
                    \PY{n}{root} \PY{o}{\PYZhy{}}\PY{o}{=} \PY{l+m+mi}{1}    
            \PY{n}{pwr} \PY{o}{+}\PY{o}{=}\PY{l+m+mi}{1}
        \PY{k}{if} \PY{n}{pwr} \PY{o}{==} \PY{l+m+mi}{6} \PY{o+ow}{and} \PY{n}{result} \PY{o}{==} \PY{k+kc}{False}\PY{p}{:}
            \PY{n+nb}{print}\PY{p}{(}\PY{l+s+s2}{\PYZdq{}}\PY{l+s+s2}{No pair of roots and powers exist.}\PY{l+s+s2}{\PYZdq{}}\PY{p}{)}
\end{Verbatim}


    \begin{Verbatim}[commandchars=\\\{\}]
Enter an integer: 26
root: 26 power: 1

    \end{Verbatim}

    \begin{Shaded}
\begin{Highlighting}[]
\CommentTok{#This doesn't account for imaginary roots as I imagine that's more complicated... pretty solid tho imo.}
\end{Highlighting}
\end{Shaded}

    \hypertarget{qquadfor-loops}{%
\subsection{\texorpdfstring{3.2\(\qquad\)For
Loops}{3.2\textbackslash{}qquadFor Loops}}\label{qquadfor-loops}}

The \texttt{while} loops we have used iterate over a sequence of
integers. Python provides a language mechanism, the \textbf{\texttt{for}
loop}, that can be used to simplify programs containing this kind of
iteration.

The general form of a \texttt{for} statement is (the words in asterisks
are desciptions of what can appear, not actual code):

\begin{Shaded}
\begin{Highlighting}[]
    \ControlFlowTok{for} \OperatorTok{*}\NormalTok{variable}\OperatorTok{*} \KeywordTok{in} \OperatorTok{*}\NormalTok{sequence}\OperatorTok{*}\NormalTok{:}
        \OperatorTok{*}\NormalTok{code block}\OperatorTok{*}
\end{Highlighting}
\end{Shaded}

The variable following \texttt{for} is assigned the first value in the
sequence, then the second value in the sequence, etc. until the sequence
is exhausted or a \textbf{break} statement is executed within the code
block.

The sequence of values bound to variable is most commonly generated
using the built-in function \textbf{range}, which returns a sequence
containing an arithmetic progression. \textbf{The \texttt{range}
function takes three integer arguments: \texttt{start,\ stop,} and
\texttt{step}.} It produces the progression:
\texttt{start,\ start\ +\ step,\ start\ +\ 2*step,} etc.\\
\textbf{If \texttt{step} is positive, the last element is the largest
integer(\texttt{start} + \texttt{i*step}) less than \texttt{stop}. If
\texttt{step} is negative, the last element is the smallest integer
greater than \texttt{stop}.}

    \hypertarget{rangestart-stop-step}{%
\subsection{\texorpdfstring{\texttt{range(start,\ stop,\ step)}}{range(start, stop, step)}}\label{rangestart-stop-step}}

    \begin{Verbatim}[commandchars=\\\{\}]
{\color{incolor}In [{\color{incolor}8}]:} \PY{n+nb}{range}\PY{p}{(}\PY{l+m+mi}{5}\PY{p}{,} \PY{l+m+mi}{40}\PY{p}{,} \PY{l+m+mi}{10}\PY{p}{)}
\end{Verbatim}


\begin{Verbatim}[commandchars=\\\{\}]
{\color{outcolor}Out[{\color{outcolor}8}]:} range(5, 40, 10)
\end{Verbatim}
            
    \emph{This doesn't output a list in Python 3, but I can confirm it does
in Python 2.}

    

    \begin{Verbatim}[commandchars=\\\{\}]
{\color{incolor}In [{\color{incolor}9}]:} \PY{n+nb}{list}\PY{p}{(}\PY{n+nb}{range}\PY{p}{(}\PY{l+m+mi}{5}\PY{p}{,}\PY{l+m+mi}{40}\PY{p}{,}\PY{l+m+mi}{10}\PY{p}{)}\PY{p}{)}
\end{Verbatim}


\begin{Verbatim}[commandchars=\\\{\}]
{\color{outcolor}Out[{\color{outcolor}9}]:} [5, 15, 25, 35]
\end{Verbatim}
            
    \begin{Shaded}
\begin{Highlighting}[]
\CommentTok{#But I guess if you do it like this, using the list function to display the range, it displays the desired code from the book.}
\end{Highlighting}
\end{Shaded}

    \begin{Verbatim}[commandchars=\\\{\}]
{\color{incolor}In [{\color{incolor}10}]:} \PY{n+nb}{list}\PY{p}{(}\PY{n+nb}{range}\PY{p}{(}\PY{l+m+mi}{40}\PY{p}{,}\PY{l+m+mi}{5}\PY{p}{,}\PY{o}{\PYZhy{}}\PY{l+m+mi}{10}\PY{p}{)}\PY{p}{)}
\end{Verbatim}


\begin{Verbatim}[commandchars=\\\{\}]
{\color{outcolor}Out[{\color{outcolor}10}]:} [40, 30, 20, 10]
\end{Verbatim}
            
    \begin{Shaded}
\begin{Highlighting}[]
\CommentTok{#In Python 3, range behaves the way xrange behaves in Python 2, so there's no need to worry about it.}
\CommentTok{#I'm not really sure what this part about specifying a sequence using a literal is about though.}
\end{Highlighting}
\end{Shaded}

    If the first argument is omitted, it defaults to 0, and if the last
argument is omitted(the step size), it defaults to 1.
\texttt{range(0,\ 3)} and \texttt{range(3)} both produce
\texttt{{[}0,\ 1,\ 2{]}}. \textbf{In other words, \texttt{start}
defaults to 0, and \texttt{step} defaults to 1.}

Less commonly, we specify the sequence to be iterated over in a
\texttt{for} loop by using a literal, e.g.(for example),
\texttt{{[}0,\ 1,\ 2{]}}.

    Think about the code:

\begin{Shaded}
\begin{Highlighting}[]
\NormalTok{    x }\OperatorTok{=} \DecValTok{4}
    \ControlFlowTok{for}\NormalTok{ i }\KeywordTok{in} \BuiltInTok{range}\NormalTok{(}\DecValTok{0}\NormalTok{, x):}
        \BuiltInTok{print}\NormalTok{(i)}
\NormalTok{        x }\OperatorTok{=} \DecValTok{5}
\end{Highlighting}
\end{Shaded}

    Having \texttt{x\ =\ 5} inside the loop here \textbf{does not} affect
the number of iterations. The \texttt{range} function in the line with
\texttt{for} is evaluated just before the first iteration of the loop,
and \textbf{not reevaluated for subsequent iterations}.

    

    \begin{Verbatim}[commandchars=\\\{\}]
{\color{incolor}In [{\color{incolor}1}]:} \PY{n}{x} \PY{o}{=} \PY{l+m+mi}{4}
        \PY{k}{for} \PY{n}{i} \PY{o+ow}{in} \PY{n+nb}{range}\PY{p}{(}\PY{l+m+mi}{0}\PY{p}{,} \PY{n}{x}\PY{p}{)}\PY{p}{:}
            \PY{n+nb}{print}\PY{p}{(}\PY{n}{i}\PY{p}{)}
            \PY{n}{x} \PY{o}{=} \PY{l+m+mi}{5}
\end{Verbatim}


    \begin{Verbatim}[commandchars=\\\{\}]
0
1
2
3

    \end{Verbatim}

    \begin{Shaded}
\begin{Highlighting}[]
\CommentTok{#This returns the same thing as if 'x = 5' was }
\CommentTok{#not in the code, so it did not affect the }
\CommentTok{#outcome at all.}
\CommentTok{#An interesting thing to note here, is that the }
\CommentTok{#variable 'i' is defined inside of the for loop}
\CommentTok{#declaration.}
\end{Highlighting}
\end{Shaded}

    

    \begin{Verbatim}[commandchars=\\\{\}]
{\color{incolor}In [{\color{incolor}12}]:} \PY{n}{x} \PY{o}{=} \PY{l+m+mi}{4}
         \PY{k}{for} \PY{n}{j} \PY{o+ow}{in} \PY{n+nb}{range}\PY{p}{(}\PY{n}{x}\PY{p}{)}\PY{p}{:}
             \PY{k}{for} \PY{n}{i} \PY{o+ow}{in} \PY{n+nb}{range}\PY{p}{(}\PY{n}{x}\PY{p}{)}\PY{p}{:}
                 \PY{n+nb}{print}\PY{p}{(}\PY{n}{i}\PY{p}{)}
                 \PY{n}{x} \PY{o}{=} \PY{l+m+mi}{2}
\end{Verbatim}


    \begin{Verbatim}[commandchars=\\\{\}]
0
1
2
3
0
1
0
1
0
1

    \end{Verbatim}

    This is the output to the above code because the \texttt{range} function
in the outer loop is only evaluated once, but the \texttt{range}
function in the inner loop is evaluated each time the inner \texttt{for}
statement is reached.

\begin{Shaded}
\begin{Highlighting}[]
\CommentTok{#You can see this is true by how the output is grouped, it's like, [0,1,2,3], [0,1], [0,1], [0,1]. So basically, the }
\CommentTok{#inner and outer loop both evaluate 4 times initially, but the inner loop, after the first outer loop, will evaluate }
\CommentTok{#using the new x value, so two times instead of 4 times. It does this 3 times because of the outer loop.}
\end{Highlighting}
\end{Shaded}

    The code below reimplements the exhaustive enumeration algorithm for
finding cube roots that we did above. This time, it'll use a
\texttt{for} loop and a \texttt{break} statement. When executed, a
\texttt{break} statement exits the innermost loop in which it is
enclosed.

    \begin{Verbatim}[commandchars=\\\{\}]
{\color{incolor}In [{\color{incolor}13}]:} \PY{c+c1}{\PYZsh{}Find the cube root of a perfect cube}
         \PY{n}{x} \PY{o}{=} \PY{n+nb}{int}\PY{p}{(}\PY{n+nb}{input}\PY{p}{(}\PY{l+s+s1}{\PYZsq{}}\PY{l+s+s1}{Enter an integer: }\PY{l+s+s1}{\PYZsq{}}\PY{p}{)}\PY{p}{)}
         \PY{k}{for} \PY{n}{ans} \PY{o+ow}{in} \PY{n+nb}{range}\PY{p}{(}\PY{l+m+mi}{0}\PY{p}{,} \PY{n+nb}{abs}\PY{p}{(}\PY{n}{x}\PY{p}{)}\PY{o}{+}\PY{l+m+mi}{1}\PY{p}{)}\PY{p}{:}
             \PY{k}{if} \PY{n}{ans}\PY{o}{*}\PY{o}{*}\PY{l+m+mi}{3} \PY{o}{\PYZgt{}}\PY{o}{=} \PY{n+nb}{abs}\PY{p}{(}\PY{n}{x}\PY{p}{)}\PY{p}{:}
                 \PY{k}{break}
         \PY{k}{if} \PY{n}{ans}\PY{o}{*}\PY{o}{*}\PY{l+m+mi}{3} \PY{o}{!=} \PY{n+nb}{abs}\PY{p}{(}\PY{n}{x}\PY{p}{)}\PY{p}{:}
             \PY{n+nb}{print}\PY{p}{(}\PY{n}{x}\PY{p}{,} \PY{l+s+s1}{\PYZsq{}}\PY{l+s+s1}{is not a perfect cube}\PY{l+s+s1}{\PYZsq{}}\PY{p}{)}
         \PY{k}{else}\PY{p}{:}
             \PY{k}{if} \PY{n}{x} \PY{o}{\PYZlt{}} \PY{l+m+mi}{0}\PY{p}{:}
                 \PY{n}{ans} \PY{o}{=} \PY{o}{\PYZhy{}}\PY{n}{ans}
             \PY{n+nb}{print}\PY{p}{(}\PY{l+s+s1}{\PYZsq{}}\PY{l+s+s1}{Cube root of}\PY{l+s+s1}{\PYZsq{}}\PY{p}{,} \PY{n}{x}\PY{p}{,}\PY{l+s+s1}{\PYZsq{}}\PY{l+s+s1}{is}\PY{l+s+s1}{\PYZsq{}}\PY{p}{,} \PY{n}{ans}\PY{p}{)}
\end{Verbatim}


    \begin{Verbatim}[commandchars=\\\{\}]
Enter an integer: 343
Cube root of 343 is 7

    \end{Verbatim}

    \begin{Verbatim}[commandchars=\\\{\}]
{\color{incolor}In [{\color{incolor}14}]:} \PY{n}{total} \PY{o}{=} \PY{l+m+mi}{0}
         \PY{k}{for} \PY{n}{c} \PY{o+ow}{in} \PY{l+s+s1}{\PYZsq{}}\PY{l+s+s1}{123456789}\PY{l+s+s1}{\PYZsq{}}\PY{p}{:}
             \PY{n}{total} \PY{o}{+}\PY{o}{=} \PY{n+nb}{int}\PY{p}{(}\PY{n}{c}\PY{p}{)}
         \PY{n+nb}{print}\PY{p}{(}\PY{n}{total}\PY{p}{)}
\end{Verbatim}


    \begin{Verbatim}[commandchars=\\\{\}]
45

    \end{Verbatim}

    \begin{Verbatim}[commandchars=\\\{\}]
{\color{incolor}In [{\color{incolor}15}]:} \PY{l+m+mi}{1}\PY{o}{+}\PY{l+m+mi}{2}\PY{o}{+}\PY{l+m+mi}{3}\PY{o}{+}\PY{l+m+mi}{4}\PY{o}{+}\PY{l+m+mi}{5}\PY{o}{+}\PY{l+m+mi}{6}\PY{o}{+}\PY{l+m+mi}{7}\PY{o}{+}\PY{l+m+mi}{8}\PY{o}{+}\PY{l+m+mi}{9}
\end{Verbatim}


\begin{Verbatim}[commandchars=\\\{\}]
{\color{outcolor}Out[{\color{outcolor}15}]:} 45
\end{Verbatim}
            
    As we can see above, the \texttt{for} statement can be used to
conveniently iterate over characters of a string. Ths above code adds up
all the numbers in the string
\texttt{\textquotesingle{}123456789\textquotesingle{}} and prints the
total.

    \textbf{Finger exercise:} Let \texttt{s} be a string that contains a
sequence of decimal numbers separated by commas, e.g.,
\texttt{s\ =\ \textquotesingle{}1.23,2.4,3.123\textquotesingle{}}. Write
a program that prints the sum of the numbers in \texttt{s}.

    \begin{Verbatim}[commandchars=\\\{\}]
{\color{incolor}In [{\color{incolor}24}]:} \PY{n}{total}\PY{p}{,} \PY{n}{s} \PY{o}{=} \PY{l+m+mi}{0}\PY{p}{,} \PY{l+s+s2}{\PYZdq{}}\PY{l+s+s2}{1.23,2.4,3.123}\PY{l+s+s2}{\PYZdq{}}
         \PY{k}{for} \PY{n}{current\PYZus{}number} \PY{o+ow}{in} \PY{n}{s}\PY{o}{.}\PY{n}{split}\PY{p}{(}\PY{l+s+s2}{\PYZdq{}}\PY{l+s+s2}{,}\PY{l+s+s2}{\PYZdq{}}\PY{p}{)}\PY{p}{:}
             \PY{n}{total} \PY{o}{+}\PY{o}{=} \PY{n+nb}{float}\PY{p}{(}\PY{n}{current\PYZus{}number}\PY{p}{)}
         \PY{n+nb}{print}\PY{p}{(}\PY{n}{total}\PY{p}{)}
\end{Verbatim}


    \begin{Verbatim}[commandchars=\\\{\}]
6.753

    \end{Verbatim}

    \begin{Verbatim}[commandchars=\\\{\}]
{\color{incolor}In [{\color{incolor}30}]:} \PY{c+c1}{\PYZsh{}n = number}
         \PY{c+c1}{\PYZsh{}l = list}
         \PY{c+c1}{\PYZsh{}s = string}
         \PY{c+c1}{\PYZsh{}f = float}
         \PY{c+c1}{\PYZsh{}e = element}
         \PY{n}{n} \PY{o}{=} \PY{n+nb}{int}\PY{p}{(}\PY{n+nb}{input}\PY{p}{(}\PY{l+s+s1}{\PYZsq{}}\PY{l+s+s1}{How many numbers?: }\PY{l+s+s1}{\PYZsq{}}\PY{p}{)}\PY{p}{)}
         \PY{n}{l} \PY{o}{=} \PY{p}{[}\PY{p}{]}
         \PY{k}{for} \PY{n}{n} \PY{o+ow}{in} \PY{n+nb}{range}\PY{p}{(}\PY{l+m+mi}{0}\PY{p}{,} \PY{n}{n}\PY{p}{)}\PY{p}{:}
             \PY{n}{l}\PY{o}{.}\PY{n}{append}\PY{p}{(}\PY{n+nb}{float}\PY{p}{(}\PY{n+nb}{input}\PY{p}{(}\PY{l+s+s2}{\PYZdq{}}\PY{l+s+s2}{Enter decimal number }\PY{l+s+s2}{\PYZdq{}} \PY{o}{+} \PY{n+nb}{str}\PY{p}{(}\PY{n}{n} \PY{o}{+} \PY{l+m+mi}{1}\PY{p}{)} \PY{o}{+} \PY{l+s+s2}{\PYZdq{}}\PY{l+s+s2}{: }\PY{l+s+s2}{\PYZdq{}}\PY{p}{)}\PY{p}{)}\PY{p}{)}
         \PY{n+nb}{sum} \PY{o}{=} \PY{l+m+mi}{0}
         \PY{n+nb}{print}\PY{p}{(}\PY{n}{l}\PY{p}{)}
         \PY{n}{s} \PY{o}{=} \PY{n}{s} \PY{o}{=} \PY{l+s+s1}{\PYZsq{}}\PY{l+s+s1}{ }\PY{l+s+s1}{\PYZsq{}}\PY{o}{.}\PY{n}{join}\PY{p}{(}\PY{n+nb}{str}\PY{p}{(}\PY{n}{e}\PY{p}{)} \PY{k}{for} \PY{n}{e} \PY{o+ow}{in} \PY{n}{l}\PY{p}{)}
         \PY{n+nb}{print}\PY{p}{(}\PY{n}{s}\PY{p}{)}
         \PY{k}{for} \PY{n}{f} \PY{o+ow}{in} \PY{n}{s}\PY{o}{.}\PY{n}{split}\PY{p}{(}\PY{l+s+s2}{\PYZdq{}}\PY{l+s+s2}{ }\PY{l+s+s2}{\PYZdq{}}\PY{p}{)}\PY{p}{:}
             \PY{n+nb}{sum} \PY{o}{+}\PY{o}{=} \PY{n+nb}{float}\PY{p}{(}\PY{n}{f}\PY{p}{)}
         \PY{n+nb}{print}\PY{p}{(}\PY{n+nb}{sum}\PY{p}{)}
\end{Verbatim}


    \begin{Verbatim}[commandchars=\\\{\}]
How many numbers?: 5
Enter decimal number 1: 1
Enter decimal number 2: 2
Enter decimal number 3: 3
Enter decimal number 4: 4
Enter decimal number 5: 5
[1.0, 2.0, 3.0, 4.0, 5.0]
1.0 2.0 3.0 4.0 5.0
15.0

    \end{Verbatim}

    \begin{Verbatim}[commandchars=\\\{\}]
{\color{incolor}In [{\color{incolor}31}]:} \PY{l+m+mi}{5}\PY{o}{+}\PY{l+m+mi}{4}\PY{o}{+}\PY{l+m+mi}{3}\PY{o}{+}\PY{l+m+mi}{2}\PY{o}{+}\PY{l+m+mi}{1}
\end{Verbatim}


\begin{Verbatim}[commandchars=\\\{\}]
{\color{outcolor}Out[{\color{outcolor}31}]:} 15
\end{Verbatim}
            
    \begin{Shaded}
\begin{Highlighting}[]
\CommentTok{#We finally did it, ugh that was a pain for no reason. I really have to remember that my variables keep their values from previous code blocks, so I have to rename them or use new variable names...}
\end{Highlighting}
\end{Shaded}

\begin{center}\rule{0.5\linewidth}{\linethickness}\end{center}

    \hypertarget{lecture-and-exercise-notesscratch-just-whatever-i-feel-like-putting-down}{%
\section{Lecture and Exercise Notes/Scratch (just whatever I feel like
putting
down)}\label{lecture-and-exercise-notesscratch-just-whatever-i-feel-like-putting-down}}

    \begin{Verbatim}[commandchars=\\\{\}]
{\color{incolor}In [{\color{incolor}25}]:} \PY{n}{hi} \PY{o}{=} \PY{l+s+s2}{\PYZdq{}}\PY{l+s+s2}{hello there}\PY{l+s+s2}{\PYZdq{}}
         \PY{n}{hi}
         \PY{n}{foo} \PY{o}{=} \PY{l+s+s2}{\PYZdq{}}\PY{l+s+s2}{this isn}\PY{l+s+s2}{\PYZsq{}}\PY{l+s+s2}{t right}\PY{l+s+s2}{\PYZdq{}}
         \PY{n}{foo}
         \PY{n}{name} \PY{o}{=} \PY{l+s+s1}{\PYZsq{}}\PY{l+s+s1}{eric}\PY{l+s+s1}{\PYZsq{}}
         \PY{n}{name}
         \PY{n}{greet} \PY{o}{=} \PY{n}{hi} \PY{o}{+} \PY{l+s+s2}{\PYZdq{}}\PY{l+s+s2}{, }\PY{l+s+s2}{\PYZdq{}} \PY{o}{+} \PY{n}{name}
         \PY{n+nb}{print}\PY{p}{(}\PY{n}{greet}\PY{p}{)}
         \PY{l+m+mi}{3}\PY{o}{*}\PY{l+s+s1}{\PYZsq{}}\PY{l+s+s1}{eric}\PY{l+s+s1}{\PYZsq{}}
         \PY{n+nb}{len}\PY{p}{(}\PY{l+s+s1}{\PYZsq{}}\PY{l+s+s1}{eric}\PY{l+s+s1}{\PYZsq{}}\PY{p}{)}
         \PY{n+nb}{len}\PY{p}{(}\PY{l+s+s1}{\PYZsq{}}\PY{l+s+s1}{hi there}\PY{l+s+s1}{\PYZsq{}}\PY{p}{)}
         \PY{l+s+s1}{\PYZsq{}}\PY{l+s+s1}{eric}\PY{l+s+s1}{\PYZsq{}}\PY{p}{[}\PY{l+m+mi}{1}\PY{p}{]}
         \PY{l+s+s1}{\PYZsq{}}\PY{l+s+s1}{eric}\PY{l+s+s1}{\PYZsq{}}\PY{p}{[}\PY{l+m+mi}{0}\PY{p}{]}
         \PY{n}{name}
         \PY{n}{name}\PY{p}{[}\PY{l+m+mi}{0}\PY{p}{]}
         \PY{l+s+s1}{\PYZsq{}}\PY{l+s+s1}{eric}\PY{l+s+s1}{\PYZsq{}}\PY{p}{[}\PY{l+m+mi}{1}\PY{p}{:}\PY{l+m+mi}{3}\PY{p}{]}
         \PY{l+s+s1}{\PYZsq{}}\PY{l+s+s1}{eric}\PY{l+s+s1}{\PYZsq{}}\PY{p}{[}\PY{p}{:}\PY{l+m+mi}{3}\PY{p}{]}
         \PY{l+s+s1}{\PYZsq{}}\PY{l+s+s1}{eric}\PY{l+s+s1}{\PYZsq{}}\PY{p}{[}\PY{l+m+mi}{1}\PY{p}{:}\PY{p}{]}
\end{Verbatim}


    \begin{Verbatim}[commandchars=\\\{\}]
hello there, eric

    \end{Verbatim}

\begin{Verbatim}[commandchars=\\\{\}]
{\color{outcolor}Out[{\color{outcolor}25}]:} 'ric'
\end{Verbatim}
            
    \begin{Verbatim}[commandchars=\\\{\}]
{\color{incolor}In [{\color{incolor} }]:} \PY{c+c1}{\PYZsh{}coding demonstration from lectures, only last output}
        \PY{c+c1}{\PYZsh{}displayed.}
\end{Verbatim}


    \begin{Verbatim}[commandchars=\\\{\}]
{\color{incolor}In [{\color{incolor}29}]:} \PY{l+s+s2}{\PYZdq{}}\PY{l+s+s2}{abcd}\PY{l+s+s2}{\PYZdq{}}\PY{p}{[}\PY{p}{:}\PY{l+m+mi}{2}\PY{p}{]}
\end{Verbatim}


\begin{Verbatim}[commandchars=\\\{\}]
{\color{outcolor}Out[{\color{outcolor}29}]:} 'ab'
\end{Verbatim}
            
    \begin{Verbatim}[commandchars=\\\{\}]
{\color{incolor}In [{\color{incolor}28}]:} \PY{l+s+s2}{\PYZdq{}}\PY{l+s+s2}{abcd}\PY{l+s+s2}{\PYZdq{}}\PY{p}{[}\PY{l+m+mi}{2}\PY{p}{:}\PY{p}{]}
\end{Verbatim}


\begin{Verbatim}[commandchars=\\\{\}]
{\color{outcolor}Out[{\color{outcolor}28}]:} 'cd'
\end{Verbatim}
            
    \begin{Verbatim}[commandchars=\\\{\}]
{\color{incolor}In [{\color{incolor}31}]:} \PY{n}{str1} \PY{o}{=} \PY{l+s+s1}{\PYZsq{}}\PY{l+s+s1}{hello}\PY{l+s+s1}{\PYZsq{}}
         \PY{n}{str1}\PY{p}{[}\PY{o}{\PYZhy{}}\PY{l+m+mi}{1}\PY{p}{]}
\end{Verbatim}


\begin{Verbatim}[commandchars=\\\{\}]
{\color{outcolor}Out[{\color{outcolor}31}]:} 'o'
\end{Verbatim}
            
    \begin{Verbatim}[commandchars=\\\{\}]
{\color{incolor}In [{\color{incolor}37}]:} \PY{n}{str4} \PY{o}{=} \PY{l+s+s1}{\PYZsq{}}\PY{l+s+s1}{helloworld}\PY{l+s+s1}{\PYZsq{}}
         \PY{n}{str4}\PY{p}{[}\PY{p}{:}\PY{o}{\PYZhy{}}\PY{l+m+mi}{1}\PY{p}{]}
\end{Verbatim}


\begin{Verbatim}[commandchars=\\\{\}]
{\color{outcolor}Out[{\color{outcolor}37}]:} 'helloworl'
\end{Verbatim}
            
    

    \begin{Verbatim}[commandchars=\\\{\}]
{\color{incolor}In [{\color{incolor}42}]:} \PY{n}{x} \PY{o}{=} \PY{l+m+mi}{1}
         \PY{n+nb}{print}\PY{p}{(}\PY{n}{x}\PY{p}{)}
         \PY{n}{x\PYZus{}str} \PY{o}{=} \PY{n+nb}{str}\PY{p}{(}\PY{n}{x}\PY{p}{)}
         \PY{n+nb}{print}\PY{p}{(}\PY{l+s+s2}{\PYZdq{}}\PY{l+s+s2}{my fav num is}\PY{l+s+s2}{\PYZdq{}}\PY{p}{,} \PY{n}{x}\PY{p}{,} \PY{l+s+s2}{\PYZdq{}}\PY{l+s+s2}{.}\PY{l+s+s2}{\PYZdq{}}\PY{p}{,} \PY{l+s+s2}{\PYZdq{}}\PY{l+s+s2}{x =}\PY{l+s+s2}{\PYZdq{}}\PY{p}{,} \PY{n}{x}\PY{p}{)}
         \PY{n+nb}{print}\PY{p}{(}\PY{l+s+s2}{\PYZdq{}}\PY{l+s+s2}{my fav num is }\PY{l+s+s2}{\PYZdq{}} \PY{o}{+} \PY{n}{x\PYZus{}str} \PY{o}{+} \PY{l+s+s2}{\PYZdq{}}\PY{l+s+s2}{. }\PY{l+s+s2}{\PYZdq{}} \PY{o}{+} \PY{l+s+s2}{\PYZdq{}}\PY{l+s+s2}{x = }\PY{l+s+s2}{\PYZdq{}} \PY{o}{+} \PY{n}{x\PYZus{}str}\PY{p}{)}
\end{Verbatim}


    \begin{Verbatim}[commandchars=\\\{\}]
1
my fav num is 1 . x = 1
my fav num is 1. x = 1

    \end{Verbatim}

    \begin{Verbatim}[commandchars=\\\{\}]
{\color{incolor}In [{\color{incolor}43}]:} \PY{n}{text} \PY{o}{=} \PY{n+nb}{input}\PY{p}{(}\PY{l+s+s2}{\PYZdq{}}\PY{l+s+s2}{type something }\PY{l+s+s2}{\PYZdq{}}\PY{p}{)}
         \PY{n+nb}{print}\PY{p}{(}\PY{l+m+mi}{5}\PY{o}{*}\PY{n}{text}\PY{p}{)}
\end{Verbatim}


    \begin{Verbatim}[commandchars=\\\{\}]
type something "foo"
"foo""foo""foo""foo""foo"

    \end{Verbatim}

    \begin{Verbatim}[commandchars=\\\{\}]
{\color{incolor}In [{\color{incolor}44}]:} \PY{n+nb}{type}\PY{p}{(}\PY{n}{text}\PY{p}{)}
\end{Verbatim}


\begin{Verbatim}[commandchars=\\\{\}]
{\color{outcolor}Out[{\color{outcolor}44}]:} str
\end{Verbatim}
            
    \begin{Shaded}
\begin{Highlighting}[]
\CommentTok{# What gets read in is automatically a string, if you want to take in a number, you have to cast it as an int or float before you can actually use it.}
\end{Highlighting}
\end{Shaded}

    \begin{Verbatim}[commandchars=\\\{\}]
{\color{incolor}In [{\color{incolor}46}]:} \PY{n}{n} \PY{o}{=} \PY{l+m+mi}{0}
         \PY{k}{while}\PY{p}{(}\PY{n}{n} \PY{o}{\PYZlt{}} \PY{l+m+mi}{5}\PY{p}{)}\PY{p}{:}
             \PY{n+nb}{print}\PY{p}{(}\PY{n}{n}\PY{p}{)}
             \PY{n}{n} \PY{o}{+}\PY{o}{=} \PY{l+m+mi}{1}
\end{Verbatim}


    \begin{Verbatim}[commandchars=\\\{\}]
0
1
2
3
4

    \end{Verbatim}

    \begin{Verbatim}[commandchars=\\\{\}]
{\color{incolor}In [{\color{incolor}47}]:} \PY{k}{for} \PY{n}{n} \PY{o+ow}{in} \PY{n+nb}{range}\PY{p}{(}\PY{l+m+mi}{5}\PY{p}{)}\PY{p}{:}
             \PY{n+nb}{print}\PY{p}{(}\PY{n}{n}\PY{p}{)}
\end{Verbatim}


    \begin{Verbatim}[commandchars=\\\{\}]
0
1
2
3
4

    \end{Verbatim}

    \begin{Verbatim}[commandchars=\\\{\}]
{\color{incolor}In [{\color{incolor}49}]:} \PY{c+c1}{\PYZsh{}simple examples while and for loop for the same thing,}
         \PY{c+c1}{\PYZsh{}showing for loop is easier to read.}
\end{Verbatim}


    \begin{Verbatim}[commandchars=\\\{\}]
{\color{incolor}In [{\color{incolor}50}]:} \PY{n}{mysum} \PY{o}{=} \PY{l+m+mi}{0}
         \PY{k}{for} \PY{n}{i} \PY{o+ow}{in} \PY{n+nb}{range}\PY{p}{(}\PY{l+m+mi}{7}\PY{p}{,}\PY{l+m+mi}{10}\PY{p}{)}\PY{p}{:}
             \PY{n}{mysum} \PY{o}{+}\PY{o}{=} \PY{n}{i}
         \PY{n+nb}{print}\PY{p}{(}\PY{n}{mysum}\PY{p}{)}
\end{Verbatim}


    \begin{Verbatim}[commandchars=\\\{\}]
24

    \end{Verbatim}

    \begin{Verbatim}[commandchars=\\\{\}]
{\color{incolor}In [{\color{incolor}51}]:} \PY{n}{mysum} \PY{o}{=} \PY{l+m+mi}{0}
         \PY{k}{for} \PY{n}{i} \PY{o+ow}{in} \PY{n+nb}{range}\PY{p}{(}\PY{l+m+mi}{5}\PY{p}{,}\PY{l+m+mi}{11}\PY{p}{,}\PY{l+m+mi}{2}\PY{p}{)}\PY{p}{:}
             \PY{n}{mysum} \PY{o}{+}\PY{o}{=} \PY{n}{i}
         \PY{n+nb}{print}\PY{p}{(}\PY{n}{mysum}\PY{p}{)}
\end{Verbatim}


    \begin{Verbatim}[commandchars=\\\{\}]
21

    \end{Verbatim}

    \begin{Verbatim}[commandchars=\\\{\}]
{\color{incolor}In [{\color{incolor}54}]:} \PY{n}{mysum} \PY{o}{=} \PY{l+m+mi}{0}
         \PY{k}{for} \PY{n}{i} \PY{o+ow}{in} \PY{n+nb}{range}\PY{p}{(}\PY{l+m+mi}{5}\PY{p}{,}\PY{l+m+mi}{11}\PY{p}{,}\PY{l+m+mi}{2}\PY{p}{)}\PY{p}{:}
             \PY{n}{mysum} \PY{o}{+}\PY{o}{=} \PY{n}{i}
             \PY{k}{if} \PY{n}{mysum} \PY{o}{==} \PY{l+m+mi}{5}\PY{p}{:}
                 \PY{k}{break}
         \PY{n+nb}{print}\PY{p}{(}\PY{n}{mysum}\PY{p}{)}
\end{Verbatim}


    \begin{Verbatim}[commandchars=\\\{\}]
5

    \end{Verbatim}

    \begin{Verbatim}[commandchars=\\\{\}]
{\color{incolor}In [{\color{incolor} }]:} \PY{c+c1}{\PYZsh{}Testing out break statement.}
\end{Verbatim}


    \begin{Verbatim}[commandchars=\\\{\}]
{\color{incolor}In [{\color{incolor}60}]:} \PY{c+c1}{\PYZsh{}You can always rewrite a for loop using a while loop.}
         \PY{c+c1}{\PYZsh{}However, you might not be able to rewrite a while loop}
         \PY{c+c1}{\PYZsh{}using a for loop.}
\end{Verbatim}


    

    \begin{Verbatim}[commandchars=\\\{\}]
{\color{incolor}In [{\color{incolor}62}]:} \PY{n}{happy} \PY{o}{=} \PY{l+m+mi}{3}
         \PY{k}{if} \PY{n}{happy} \PY{o}{\PYZgt{}} \PY{l+m+mi}{2}\PY{p}{:}
             \PY{n+nb}{print}\PY{p}{(}\PY{l+s+s1}{\PYZsq{}}\PY{l+s+s1}{hello world}\PY{l+s+s1}{\PYZsq{}}\PY{p}{)}
\end{Verbatim}


    \begin{Verbatim}[commandchars=\\\{\}]
hello world

    \end{Verbatim}

    \begin{Verbatim}[commandchars=\\\{\}]
{\color{incolor}In [{\color{incolor}63}]:} \PY{c+c1}{\PYZsh{}Please remember to end your if statements }
         \PY{c+c1}{\PYZsh{}and other statements with :}
         \PY{c+c1}{\PYZsh{}I always forget.}
\end{Verbatim}


    \begin{Verbatim}[commandchars=\\\{\}]
{\color{incolor}In [{\color{incolor}70}]:} \PY{c+c1}{\PYZsh{}vara varb exercise}
         \PY{n}{varA} \PY{o}{=} \PY{l+m+mi}{32323}
         \PY{n}{varB} \PY{o}{=} \PY{l+s+s2}{\PYZdq{}}\PY{l+s+s2}{string}\PY{l+s+s2}{\PYZdq{}}
         \PY{k}{if} \PY{n+nb}{type}\PY{p}{(}\PY{n}{varA}\PY{p}{)} \PY{o}{==} \PY{n+nb}{str} \PY{o+ow}{or} \PY{n+nb}{type}\PY{p}{(}\PY{n}{varB}\PY{p}{)} \PY{o}{==} \PY{n+nb}{str}\PY{p}{:}
             \PY{n+nb}{print}\PY{p}{(}\PY{l+s+s2}{\PYZdq{}}\PY{l+s+s2}{string involved}\PY{l+s+s2}{\PYZdq{}}\PY{p}{)}
         \PY{k}{elif} \PY{n}{varA} \PY{o}{\PYZgt{}} \PY{n}{varB}\PY{p}{:}
             \PY{n+nb}{print}\PY{p}{(}\PY{l+s+s2}{\PYZdq{}}\PY{l+s+s2}{bigger}\PY{l+s+s2}{\PYZdq{}}\PY{p}{)}
         \PY{k}{elif} \PY{n}{varA} \PY{o}{==} \PY{n}{varB}\PY{p}{:}
             \PY{n+nb}{print}\PY{p}{(}\PY{l+s+s2}{\PYZdq{}}\PY{l+s+s2}{equal}\PY{l+s+s2}{\PYZdq{}}\PY{p}{)}
         \PY{k}{elif} \PY{n}{varA} \PY{o}{\PYZlt{}} \PY{n}{varB}\PY{p}{:}
             \PY{n+nb}{print}\PY{p}{(}\PY{l+s+s2}{\PYZdq{}}\PY{l+s+s2}{smaller}\PY{l+s+s2}{\PYZdq{}}\PY{p}{)}
\end{Verbatim}


    \begin{Verbatim}[commandchars=\\\{\}]
string involved

    \end{Verbatim}

    \begin{Verbatim}[commandchars=\\\{\}]
{\color{incolor}In [{\color{incolor} }]:} \PY{c+c1}{\PYZsh{}You can stop an infinite loop in your program by typing CTRL+c in the console.}
\end{Verbatim}


    \begin{Verbatim}[commandchars=\\\{\}]
{\color{incolor}In [{\color{incolor}71}]:} \PY{c+c1}{\PYZsh{}You can purposely use an infinite loops like this:}
         \PY{k}{while} \PY{k+kc}{True}\PY{p}{:}
             \PY{c+c1}{\PYZsh{}And then break out of it using \PYZdq{}break\PYZdq{}}
             \PY{k}{break}
\end{Verbatim}


    \begin{Verbatim}[commandchars=\\\{\}]
{\color{incolor}In [{\color{incolor} }]:} \PY{c+c1}{\PYZsh{}This also works.}
        \PY{k}{while} \PY{o+ow}{not} \PY{k+kc}{False}\PY{p}{:}
            \PY{k}{break}
\end{Verbatim}


    

    \begin{Verbatim}[commandchars=\\\{\}]
{\color{incolor}In [{\color{incolor}73}]:} \PY{n}{num} \PY{o}{=} \PY{l+m+mi}{10}
         \PY{k}{while} \PY{k+kc}{True}\PY{p}{:}
             \PY{k}{if} \PY{n}{num} \PY{o}{\PYZlt{}} \PY{l+m+mi}{7}\PY{p}{:}
                 \PY{n+nb}{print}\PY{p}{(}\PY{l+s+s1}{\PYZsq{}}\PY{l+s+s1}{Breaking out of loop}\PY{l+s+s1}{\PYZsq{}}\PY{p}{)}
                 \PY{k}{break}
             \PY{n+nb}{print}\PY{p}{(}\PY{n}{num}\PY{p}{)}
             \PY{n}{num} \PY{o}{\PYZhy{}}\PY{o}{=} \PY{l+m+mi}{1}
         \PY{n+nb}{print}\PY{p}{(}\PY{l+s+s1}{\PYZsq{}}\PY{l+s+s1}{Outside of loop}\PY{l+s+s1}{\PYZsq{}}\PY{p}{)}
\end{Verbatim}


    \begin{Verbatim}[commandchars=\\\{\}]
10
9
8
7
Breaking out of loop
Outside of loop

    \end{Verbatim}

    \begin{Verbatim}[commandchars=\\\{\}]
{\color{incolor}In [{\color{incolor}3}]:} \PY{c+c1}{\PYZsh{}alright fine this program was a bit overdone}
        \PY{n}{i} \PY{o}{=} \PY{l+m+mi}{0}
        \PY{k}{while}\PY{p}{(}\PY{n}{i} \PY{o}{\PYZlt{}}\PY{o}{=} \PY{l+m+mi}{10}\PY{p}{)}\PY{p}{:}
            \PY{k}{if} \PY{n}{i} \PY{o}{==} \PY{l+m+mi}{2}\PY{p}{:}
                \PY{n+nb}{print}\PY{p}{(}\PY{l+s+s2}{\PYZdq{}}\PY{l+s+s2}{print }\PY{l+s+s2}{\PYZdq{}}\PY{p}{,} \PY{n}{i}\PY{p}{)}
                \PY{n}{i} \PY{o}{+}\PY{o}{=} \PY{l+m+mi}{1}
            \PY{k}{elif} \PY{n}{i} \PY{o}{\PYZpc{}} \PY{l+m+mi}{2} \PY{o}{==} \PY{l+m+mi}{0} \PY{o+ow}{and} \PY{n}{i} \PY{o}{!=} \PY{l+m+mi}{0}\PY{p}{:}
                \PY{n+nb}{print}\PY{p}{(}\PY{l+s+s2}{\PYZdq{}}\PY{l+s+s2}{prints }\PY{l+s+s2}{\PYZdq{}}\PY{p}{,} \PY{n}{i}\PY{p}{)}
                \PY{n}{i} \PY{o}{+}\PY{o}{=} \PY{l+m+mi}{1}
            \PY{k}{else}\PY{p}{:}
                \PY{n}{i} \PY{o}{+}\PY{o}{=} \PY{l+m+mi}{1}
        \PY{n+nb}{print}\PY{p}{(}\PY{l+s+s2}{\PYZdq{}}\PY{l+s+s2}{prints Goodbye!}\PY{l+s+s2}{\PYZdq{}}\PY{p}{)}
\end{Verbatim}


    \begin{Verbatim}[commandchars=\\\{\}]
print  2
prints  4
prints  6
prints  8
prints  10
prints Goodbye!

    \end{Verbatim}

    \begin{Verbatim}[commandchars=\\\{\}]
{\color{incolor}In [{\color{incolor}8}]:} \PY{c+c1}{\PYZsh{}Yeah woops I guess I took it too literally}
        \PY{c+c1}{\PYZsh{}This is correct.}
        \PY{n}{i} \PY{o}{=} \PY{l+m+mi}{2}
        \PY{k}{while} \PY{n}{i} \PY{o}{\PYZlt{}}\PY{o}{=} \PY{l+m+mi}{10}\PY{p}{:}
            \PY{n+nb}{print}\PY{p}{(}\PY{n}{i}\PY{p}{)}
            \PY{n}{i} \PY{o}{+}\PY{o}{=} \PY{l+m+mi}{2}
        \PY{n+nb}{print}\PY{p}{(}\PY{l+s+s2}{\PYZdq{}}\PY{l+s+s2}{Goodbye!}\PY{l+s+s2}{\PYZdq{}}\PY{p}{)}
\end{Verbatim}


    \begin{Verbatim}[commandchars=\\\{\}]
2
4
6
8
10
Goodbye!

    \end{Verbatim}

    \begin{Verbatim}[commandchars=\\\{\}]
{\color{incolor}In [{\color{incolor}11}]:} \PY{n}{i} \PY{o}{=} \PY{l+m+mi}{10}
         \PY{n+nb}{print}\PY{p}{(}\PY{l+s+s2}{\PYZdq{}}\PY{l+s+s2}{Hello!}\PY{l+s+s2}{\PYZdq{}}\PY{p}{)}
         \PY{k}{while} \PY{n}{i} \PY{o}{\PYZgt{}}\PY{o}{=} \PY{l+m+mi}{2}\PY{p}{:}
             \PY{n+nb}{print}\PY{p}{(}\PY{n}{i}\PY{p}{)}
             \PY{n}{i} \PY{o}{\PYZhy{}}\PY{o}{=} \PY{l+m+mi}{2}
\end{Verbatim}


    \begin{Verbatim}[commandchars=\\\{\}]
Hello!
10
8
6
4
2

    \end{Verbatim}

    

    \begin{Verbatim}[commandchars=\\\{\}]
{\color{incolor}In [{\color{incolor}26}]:} \PY{c+c1}{\PYZsh{}Well, I thought this wouldn\PYZsq{}t be accepted.}
         \PY{c+c1}{\PYZsh{}Can\PYZsq{}t use a for loop or range.}
         \PY{c+c1}{\PYZsh{}I could reuse this for the for loop part though!}
         \PY{n}{end} \PY{o}{=} \PY{l+m+mi}{250}
         \PY{n}{summation} \PY{o}{=} \PY{l+m+mi}{0}
         \PY{k}{for} \PY{n}{i} \PY{o+ow}{in} \PY{n+nb}{range}\PY{p}{(}\PY{l+m+mi}{1}\PY{p}{,} \PY{n}{end} \PY{o}{+} \PY{l+m+mi}{1}\PY{p}{)}\PY{p}{:}
             \PY{n}{summation} \PY{o}{+}\PY{o}{=} \PY{n}{i}
             \PY{c+c1}{\PYZsh{}print(i)}
             \PY{c+c1}{\PYZsh{}print(summation)}
         \PY{n+nb}{print}\PY{p}{(}\PY{n}{summation}\PY{p}{)}
\end{Verbatim}


    \begin{Verbatim}[commandchars=\\\{\}]
31375

    \end{Verbatim}

    \begin{Verbatim}[commandchars=\\\{\}]
{\color{incolor}In [{\color{incolor}33}]:} \PY{n}{end} \PY{o}{=} \PY{l+m+mi}{6}
         \PY{n}{i} \PY{o}{=} \PY{l+m+mi}{0}
         \PY{n}{summation} \PY{o}{=} \PY{l+m+mi}{0}
         \PY{k}{while} \PY{n}{i} \PY{o}{\PYZlt{}} \PY{n}{end}\PY{p}{:}
             \PY{n}{i} \PY{o}{+}\PY{o}{=} \PY{l+m+mi}{1}
             \PY{n}{summation} \PY{o}{+}\PY{o}{=} \PY{n}{i}
         \PY{n+nb}{print}\PY{p}{(}\PY{n}{summation}\PY{p}{)}
\end{Verbatim}


    \begin{Verbatim}[commandchars=\\\{\}]
21

    \end{Verbatim}

    

    \begin{Verbatim}[commandchars=\\\{\}]
{\color{incolor}In [{\color{incolor}36}]:} \PY{k}{for} \PY{n}{a} \PY{o+ow}{in} \PY{n+nb}{range}\PY{p}{(}\PY{l+m+mi}{2}\PY{p}{,}\PY{l+m+mi}{11}\PY{p}{,}\PY{l+m+mi}{2}\PY{p}{)}\PY{p}{:}
             \PY{n+nb}{print}\PY{p}{(}\PY{n}{a}\PY{p}{)}
         \PY{n+nb}{print}\PY{p}{(}\PY{l+s+s2}{\PYZdq{}}\PY{l+s+s2}{Goodbye!}\PY{l+s+s2}{\PYZdq{}}\PY{p}{)}
\end{Verbatim}


    \begin{Verbatim}[commandchars=\\\{\}]
2
4
6
8
10
Goodbye!

    \end{Verbatim}

    \begin{Verbatim}[commandchars=\\\{\}]
{\color{incolor}In [{\color{incolor}1}]:} \PY{n+nb}{print}\PY{p}{(}\PY{l+s+s2}{\PYZdq{}}\PY{l+s+s2}{Hello!}\PY{l+s+s2}{\PYZdq{}}\PY{p}{)}
        \PY{k}{for} \PY{n}{i} \PY{o+ow}{in} \PY{n+nb}{range}\PY{p}{(}\PY{l+m+mi}{10}\PY{p}{,}\PY{l+m+mi}{0}\PY{p}{,}\PY{o}{\PYZhy{}}\PY{l+m+mi}{2}\PY{p}{)}\PY{p}{:}
            \PY{n+nb}{print}\PY{p}{(}\PY{n}{i}\PY{p}{)}
\end{Verbatim}


    \begin{Verbatim}[commandchars=\\\{\}]
Hello!
10
8
6
4
2

    \end{Verbatim}

    \begin{Verbatim}[commandchars=\\\{\}]
{\color{incolor}In [{\color{incolor}3}]:} \PY{n}{x} \PY{o}{=} \PY{l+m+mi}{5}
        \PY{n}{ans} \PY{o}{=} \PY{l+m+mi}{0}
        \PY{n}{itersLeft} \PY{o}{=} \PY{n}{x}
        \PY{k}{while}\PY{p}{(}\PY{n}{itersLeft} \PY{o}{!=} \PY{l+m+mi}{0}\PY{p}{)}\PY{p}{:}
            \PY{n}{ans} \PY{o}{=} \PY{n}{ans} \PY{o}{+} \PY{n}{x}
            \PY{n}{itersLeft} \PY{o}{=} \PY{n}{itersLeft} \PY{o}{\PYZhy{}} \PY{l+m+mi}{1}
        \PY{n+nb}{print}\PY{p}{(}\PY{n+nb}{str}\PY{p}{(}\PY{n}{x}\PY{p}{)} \PY{o}{+} \PY{l+s+s1}{\PYZsq{}}\PY{l+s+s1}{*}\PY{l+s+s1}{\PYZsq{}} \PY{o}{+} \PY{n+nb}{str}\PY{p}{(}\PY{n}{x}\PY{p}{)} \PY{o}{+} \PY{l+s+s1}{\PYZsq{}}\PY{l+s+s1}{ = }\PY{l+s+s1}{\PYZsq{}} \PY{o}{+} \PY{n+nb}{str}\PY{p}{(}\PY{n}{ans}\PY{p}{)}\PY{p}{)}
\end{Verbatim}


    \begin{Verbatim}[commandchars=\\\{\}]
5*5 = 25

    \end{Verbatim}

    \begin{Verbatim}[commandchars=\\\{\}]
{\color{incolor}In [{\color{incolor}5}]:} \PY{c+c1}{\PYZsh{}don\PYZsq{}t be fooled here, num is redfined}
        \PY{c+c1}{\PYZsh{}in the for loop.}
        \PY{n}{num} \PY{o}{=} \PY{l+m+mi}{10}
        \PY{k}{for} \PY{n}{num} \PY{o+ow}{in} \PY{n+nb}{range}\PY{p}{(}\PY{l+m+mi}{5}\PY{p}{)}\PY{p}{:}
            \PY{n+nb}{print}\PY{p}{(}\PY{n}{num}\PY{p}{)}
        \PY{n+nb}{print}\PY{p}{(}\PY{n}{num}\PY{p}{)}
\end{Verbatim}


    \begin{Verbatim}[commandchars=\\\{\}]
0
1
2
3
4
4

    \end{Verbatim}

    \begin{Verbatim}[commandchars=\\\{\}]
{\color{incolor}In [{\color{incolor}6}]:} \PY{c+c1}{\PYZsh{}division always gives a float}
        \PY{c+c1}{\PYZsh{}for some reason dividing 0 by a number gives 0}
        \PY{c+c1}{\PYZsh{}maybe im just dumb}
        \PY{n}{divisor} \PY{o}{=} \PY{l+m+mi}{2}
        \PY{k}{for} \PY{n}{num} \PY{o+ow}{in} \PY{n+nb}{range}\PY{p}{(}\PY{l+m+mi}{0}\PY{p}{,} \PY{l+m+mi}{10}\PY{p}{,} \PY{l+m+mi}{2}\PY{p}{)}\PY{p}{:}
                \PY{n+nb}{print}\PY{p}{(}\PY{n}{num}\PY{o}{/}\PY{n}{divisor}\PY{p}{)}
\end{Verbatim}


    \begin{Verbatim}[commandchars=\\\{\}]
0.0
1.0
2.0
3.0
4.0

    \end{Verbatim}

    \begin{Verbatim}[commandchars=\\\{\}]
{\color{incolor}In [{\color{incolor}7}]:} \PY{k}{for} \PY{n}{variable} \PY{o+ow}{in} \PY{n+nb}{range}\PY{p}{(}\PY{l+m+mi}{20}\PY{p}{)}\PY{p}{:}
            \PY{k}{if} \PY{n}{variable} \PY{o}{\PYZpc{}} \PY{l+m+mi}{4} \PY{o}{==} \PY{l+m+mi}{0}\PY{p}{:}
                \PY{n+nb}{print}\PY{p}{(}\PY{n}{variable}\PY{p}{)}
            \PY{k}{if} \PY{n}{variable} \PY{o}{\PYZpc{}} \PY{l+m+mi}{16} \PY{o}{==} \PY{l+m+mi}{0}\PY{p}{:}
                \PY{n+nb}{print}\PY{p}{(}\PY{l+s+s1}{\PYZsq{}}\PY{l+s+s1}{Foo!}\PY{l+s+s1}{\PYZsq{}}\PY{p}{)} 
\end{Verbatim}


    \begin{Verbatim}[commandchars=\\\{\}]
0
Foo!
4
8
12
16
Foo!

    \end{Verbatim}

    \begin{Verbatim}[commandchars=\\\{\}]
{\color{incolor}In [{\color{incolor}10}]:} \PY{c+c1}{\PYZsh{} o never prints out because it\PYZsq{}s a vowel and gets parsed in the first if statement.}
         \PY{c+c1}{\PYZsh{} I missed the dumb spaces for the value of numCons, I had \PYZhy{}22 but there are 3 spaces so it\PYZsq{}s \PYZhy{}25.}
         \PY{n}{school} \PY{o}{=} \PY{l+s+s1}{\PYZsq{}}\PY{l+s+s1}{Massachusetts Institute of Technology}\PY{l+s+s1}{\PYZsq{}}
         \PY{n}{numVowels} \PY{o}{=} \PY{l+m+mi}{0}
         \PY{n}{numCons} \PY{o}{=} \PY{l+m+mi}{0}
         \PY{n}{cons} \PY{o}{=} \PY{p}{[}\PY{p}{]}
         
         \PY{k}{for} \PY{n}{char} \PY{o+ow}{in} \PY{n}{school}\PY{p}{:}
             \PY{k}{if} \PY{n}{char} \PY{o}{==} \PY{l+s+s1}{\PYZsq{}}\PY{l+s+s1}{a}\PY{l+s+s1}{\PYZsq{}} \PY{o+ow}{or} \PY{n}{char} \PY{o}{==} \PY{l+s+s1}{\PYZsq{}}\PY{l+s+s1}{e}\PY{l+s+s1}{\PYZsq{}} \PY{o+ow}{or} \PY{n}{char} \PY{o}{==} \PY{l+s+s1}{\PYZsq{}}\PY{l+s+s1}{i}\PY{l+s+s1}{\PYZsq{}} \PYZbs{}
                \PY{o+ow}{or} \PY{n}{char} \PY{o}{==} \PY{l+s+s1}{\PYZsq{}}\PY{l+s+s1}{o}\PY{l+s+s1}{\PYZsq{}} \PY{o+ow}{or} \PY{n}{char} \PY{o}{==} \PY{l+s+s1}{\PYZsq{}}\PY{l+s+s1}{u}\PY{l+s+s1}{\PYZsq{}}\PY{p}{:}
                 \PY{n}{numVowels} \PY{o}{+}\PY{o}{=} \PY{l+m+mi}{1}
             \PY{k}{elif} \PY{n}{char} \PY{o}{==} \PY{l+s+s1}{\PYZsq{}}\PY{l+s+s1}{o}\PY{l+s+s1}{\PYZsq{}} \PY{o+ow}{or} \PY{n}{char} \PY{o}{==} \PY{l+s+s1}{\PYZsq{}}\PY{l+s+s1}{M}\PY{l+s+s1}{\PYZsq{}}\PY{p}{:}
                 \PY{n+nb}{print}\PY{p}{(}\PY{n}{char}\PY{p}{)}
             \PY{k}{else}\PY{p}{:}
                 \PY{n}{cons}\PY{o}{.}\PY{n}{append}\PY{p}{(}\PY{n}{char}\PY{p}{)}
                 \PY{n}{numCons} \PY{o}{\PYZhy{}}\PY{o}{=} \PY{l+m+mi}{1}
         
         \PY{n+nb}{print}\PY{p}{(}\PY{l+s+s1}{\PYZsq{}}\PY{l+s+s1}{numVowels is: }\PY{l+s+s1}{\PYZsq{}} \PY{o}{+} \PY{n+nb}{str}\PY{p}{(}\PY{n}{numVowels}\PY{p}{)}\PY{p}{)}
         \PY{n+nb}{print}\PY{p}{(}\PY{l+s+s1}{\PYZsq{}}\PY{l+s+s1}{numCons is: }\PY{l+s+s1}{\PYZsq{}} \PY{o}{+} \PY{n+nb}{str}\PY{p}{(}\PY{n}{numCons}\PY{p}{)}\PY{p}{)} 
         \PY{n+nb}{print}\PY{p}{(}\PY{n}{cons}\PY{p}{)}
\end{Verbatim}


    \begin{Verbatim}[commandchars=\\\{\}]
M
numVowels is: 11
numCons is: -25
['s', 's', 'c', 'h', 's', 't', 't', 's', ' ', 'I', 'n', 's', 't', 't', 't', ' ', 'f', ' ', 'T', 'c', 'h', 'n', 'l', 'g', 'y']

    \end{Verbatim}

    \begin{Verbatim}[commandchars=\\\{\}]
{\color{incolor}In [{\color{incolor} }]:} \PY{c+c1}{\PYZsh{} Make sure to always define/initialize the variable before the loop, when using loops.}
\end{Verbatim}


    \begin{Verbatim}[commandchars=\\\{\}]
{\color{incolor}In [{\color{incolor}15}]:} \PY{c+c1}{\PYZsh{} Cleaner Guess and Check}
         \PY{c+c1}{\PYZsh{} cube root}
         \PY{n}{cube} \PY{o}{=} \PY{n+nb}{int}\PY{p}{(}\PY{n+nb}{input}\PY{p}{(}\PY{l+s+s2}{\PYZdq{}}\PY{l+s+s2}{Enter an Integer: }\PY{l+s+s2}{\PYZdq{}}\PY{p}{)}\PY{p}{)}
         \PY{k}{for} \PY{n}{guess} \PY{o+ow}{in} \PY{n+nb}{range}\PY{p}{(}\PY{n+nb}{abs}\PY{p}{(}\PY{n}{cube}\PY{p}{)}\PY{o}{+}\PY{l+m+mi}{1}\PY{p}{)}\PY{p}{:}
             \PY{k}{if} \PY{n}{guess}\PY{o}{*}\PY{o}{*}\PY{l+m+mi}{3} \PY{o}{==} \PY{n+nb}{abs}\PY{p}{(}\PY{n}{cube}\PY{p}{)}\PY{p}{:}
                 \PY{k}{break}
         \PY{k}{if} \PY{n}{guess}\PY{o}{*}\PY{o}{*}\PY{l+m+mi}{3} \PY{o}{!=} \PY{n+nb}{abs}\PY{p}{(}\PY{n}{cube}\PY{p}{)}\PY{p}{:}
             \PY{n+nb}{print}\PY{p}{(}\PY{n}{cube}\PY{p}{,} \PY{l+s+s2}{\PYZdq{}}\PY{l+s+s2}{is not a perfect cube.}\PY{l+s+s2}{\PYZdq{}}\PY{p}{)}
         \PY{k}{else}\PY{p}{:}
             \PY{k}{if} \PY{n}{cube} \PY{o}{\PYZlt{}} \PY{l+m+mi}{0}\PY{p}{:}
                 \PY{n}{guess} \PY{o}{=} \PY{o}{\PYZhy{}}\PY{n}{guess}
                 
             \PY{n+nb}{print}\PY{p}{(}\PY{l+s+s2}{\PYZdq{}}\PY{l+s+s2}{Cube root of }\PY{l+s+s2}{\PYZdq{}}\PY{p}{,} \PY{n}{cube}\PY{p}{,} \PY{l+s+s2}{\PYZdq{}}\PY{l+s+s2}{ is }\PY{l+s+s2}{\PYZdq{}}\PY{p}{,} \PY{n}{guess}\PY{p}{)}
\end{Verbatim}


    \begin{Verbatim}[commandchars=\\\{\}]
Enter an Integer: -28
-28 is not a perfect cube.

    \end{Verbatim}

    \begin{Verbatim}[commandchars=\\\{\}]
{\color{incolor}In [{\color{incolor}19}]:} \PY{l+s+sd}{\PYZsq{}\PYZsq{}\PYZsq{}You shouldn\PYZsq{}t just deal with cases where you get what you expect. You should prepare for other cases to see if your code does the right thing.\PYZsq{}\PYZsq{}\PYZsq{}}
\end{Verbatim}


\begin{Verbatim}[commandchars=\\\{\}]
{\color{outcolor}Out[{\color{outcolor}19}]:} "You shouldn't just deal with cases where you get what you expect. You should prepare for other cases to see if your code does the right thing."
\end{Verbatim}
            
    \begin{Verbatim}[commandchars=\\\{\}]
{\color{incolor}In [{\color{incolor}22}]:} \PY{c+c1}{\PYZsh{}PSET 1 Problem 1}
         \PY{n}{s} \PY{o}{=} \PY{l+s+s1}{\PYZsq{}}\PY{l+s+s1}{azcbobobegghakl}\PY{l+s+s1}{\PYZsq{}}
         \PY{n}{numVowels} \PY{o}{=} \PY{l+m+mi}{0}
         \PY{k}{for} \PY{n}{char} \PY{o+ow}{in} \PY{n}{s}\PY{p}{:}
             \PY{k}{if} \PY{n}{char} \PY{o}{==} \PY{l+s+s1}{\PYZsq{}}\PY{l+s+s1}{a}\PY{l+s+s1}{\PYZsq{}} \PY{o+ow}{or} \PY{n}{char} \PY{o}{==} \PY{l+s+s1}{\PYZsq{}}\PY{l+s+s1}{e}\PY{l+s+s1}{\PYZsq{}} \PY{o+ow}{or} \PY{n}{char} \PY{o}{==} \PY{l+s+s1}{\PYZsq{}}\PY{l+s+s1}{i}\PY{l+s+s1}{\PYZsq{}} \PY{o+ow}{or} \PY{n}{char} \PY{o}{==} \PY{l+s+s1}{\PYZsq{}}\PY{l+s+s1}{o}\PY{l+s+s1}{\PYZsq{}} \PY{o+ow}{or} \PY{n}{char} \PY{o}{==} \PY{l+s+s1}{\PYZsq{}}\PY{l+s+s1}{u}\PY{l+s+s1}{\PYZsq{}}\PY{p}{:}
                 \PY{n}{numVowels} \PY{o}{+}\PY{o}{=} \PY{l+m+mi}{1}
         \PY{n+nb}{print}\PY{p}{(}\PY{l+s+s2}{\PYZdq{}}\PY{l+s+s2}{Number of vowels: }\PY{l+s+s2}{\PYZdq{}} \PY{o}{+} \PY{n+nb}{str}\PY{p}{(}\PY{n}{numVowels}\PY{p}{)}\PY{p}{)}
\end{Verbatim}


    \begin{Verbatim}[commandchars=\\\{\}]
Number of vowels: 5

    \end{Verbatim}

    \begin{Verbatim}[commandchars=\\\{\}]
{\color{incolor}In [{\color{incolor}74}]:} \PY{c+c1}{\PYZsh{}PSET 1 Problem 2}
         \PY{n}{s} \PY{o}{=} \PY{l+s+s1}{\PYZsq{}}\PY{l+s+s1}{bobobobobobo}\PY{l+s+s1}{\PYZsq{}}
         \PY{n}{numBob} \PY{o}{=} \PY{l+m+mi}{0}
         \PY{n}{testBob} \PY{o}{=} \PY{l+s+s1}{\PYZsq{}}\PY{l+s+s1}{\PYZsq{}}
         \PY{k}{for} \PY{n}{char} \PY{o+ow}{in} \PY{n}{s}\PY{p}{:}
             \PY{n}{testBob} \PY{o}{+}\PY{o}{=} \PY{n}{char}
             \PY{k}{if} \PY{n+nb}{len}\PY{p}{(}\PY{n}{testBob}\PY{p}{)} \PY{o}{==} \PY{l+m+mi}{3}\PY{p}{:}
                 \PY{k}{if} \PY{n}{testBob} \PY{o}{==} \PY{l+s+s1}{\PYZsq{}}\PY{l+s+s1}{bob}\PY{l+s+s1}{\PYZsq{}}\PY{p}{:}
                     \PY{n}{numBob} \PY{o}{+}\PY{o}{=} \PY{l+m+mi}{1}
             \PY{k}{if} \PY{n+nb}{len}\PY{p}{(}\PY{n}{testBob}\PY{p}{)} \PY{o}{\PYZgt{}} \PY{l+m+mi}{2}\PY{p}{:}
                 \PY{n}{testBob} \PY{o}{=} \PY{n}{testBob}\PY{p}{[}\PY{l+m+mi}{1}\PY{p}{:}\PY{p}{]}
         \PY{n+nb}{print}\PY{p}{(}\PY{l+s+s2}{\PYZdq{}}\PY{l+s+s2}{Number of times bob occurs is: }\PY{l+s+s2}{\PYZdq{}} \PY{o}{+} \PY{n+nb}{str}\PY{p}{(}\PY{n}{numBob}\PY{p}{)}\PY{p}{)}
\end{Verbatim}


    \begin{Verbatim}[commandchars=\\\{\}]
Number of times bob occurs is: 5

    \end{Verbatim}

    \begin{Verbatim}[commandchars=\\\{\}]
{\color{incolor}In [{\color{incolor}221}]:} \PY{c+c1}{\PYZsh{}PSET 1 Problem 3}
          \PY{c+c1}{\PYZsh{}This was actually so hard. Easy once I figured out that comparison operands(\PYZgt{}, \PYZlt{}, etc) work for strings. (a \PYZlt{} b == True etc)}
          \PY{n}{s} \PY{o}{=} \PY{l+s+s1}{\PYZsq{}}\PY{l+s+s1}{abcbcd}\PY{l+s+s1}{\PYZsq{}}
          \PY{n}{alphabet} \PY{o}{=} \PY{l+s+s1}{\PYZsq{}}\PY{l+s+s1}{abcdefghijklmnopqrstuvwxyz}\PY{l+s+s1}{\PYZsq{}}
          \PY{n}{i} \PY{o}{=} \PY{l+m+mi}{0}
          \PY{n}{largestString} \PY{o}{=} \PY{l+s+s1}{\PYZsq{}}\PY{l+s+s1}{\PYZsq{}}
          \PY{k}{while} \PY{n}{i} \PY{o}{\PYZlt{}} \PY{n+nb}{len}\PY{p}{(}\PY{n}{s}\PY{p}{)}\PY{p}{:}
              \PY{n}{subString} \PY{o}{=} \PY{l+s+s1}{\PYZsq{}}\PY{l+s+s1}{\PYZsq{}}
              \PY{n}{test} \PY{o}{=} \PY{n}{s}\PY{p}{[}\PY{n}{i}\PY{p}{:}\PY{p}{]}
              \PY{n}{i} \PY{o}{+}\PY{o}{=} \PY{l+m+mi}{1}
              \PY{k}{for} \PY{n}{char} \PY{o+ow}{in} \PY{n}{test}\PY{p}{:}
                  \PY{k}{if} \PY{n+nb}{len}\PY{p}{(}\PY{n}{subString}\PY{p}{)} \PY{o}{==} \PY{l+m+mi}{0}\PY{p}{:}
                      \PY{n}{subString} \PY{o}{+}\PY{o}{=} \PY{n}{char}
                  \PY{k}{elif} \PY{n}{char} \PY{o}{\PYZgt{}}\PY{o}{=} \PY{n}{subString}\PY{p}{[}\PY{o}{\PYZhy{}}\PY{l+m+mi}{1}\PY{p}{]}\PY{p}{:}
                      \PY{n}{subString} \PY{o}{+}\PY{o}{=} \PY{n}{char}
                  \PY{k}{elif} \PY{n}{char} \PY{o}{\PYZlt{}} \PY{n}{subString}\PY{p}{[}\PY{o}{\PYZhy{}}\PY{l+m+mi}{1}\PY{p}{]}\PY{p}{:}
                      \PY{k}{break}
                  \PY{c+c1}{\PYZsh{}print(subString)}
              \PY{c+c1}{\PYZsh{}if subString \PYZgt{} largestString, then if lengths are tied, largestString doesn\PYZsq{}t get replaced.}
              \PY{c+c1}{\PYZsh{}if subString \PYZgt{}= largestString, then if lengths are tied, largestString does get replaced.}
              \PY{c+c1}{\PYZsh{}we want to print the first substring in ties so we don\PYZsq{}t want it to get replaced.}
              \PY{k}{if} \PY{n+nb}{len}\PY{p}{(}\PY{n}{subString}\PY{p}{)} \PY{o}{\PYZgt{}} \PY{n+nb}{len}\PY{p}{(}\PY{n}{largestString}\PY{p}{)}\PY{p}{:}
                  \PY{n}{largestString} \PY{o}{=} \PY{n}{subString}
          
          \PY{n+nb}{print}\PY{p}{(}\PY{l+s+s2}{\PYZdq{}}\PY{l+s+s2}{Longest substring in alphabetical order is: }\PY{l+s+s2}{\PYZdq{}} \PY{o}{+} \PY{n}{largestString}\PY{p}{)}
                  
\end{Verbatim}


    \begin{Verbatim}[commandchars=\\\{\}]
Longest substring in alphabetical order is: beggh

    \end{Verbatim}

    \begin{Verbatim}[commandchars=\\\{\}]
{\color{incolor}In [{\color{incolor}190}]:} \PY{l+s+s1}{\PYZsq{}}\PY{l+s+s1}{a}\PY{l+s+s1}{\PYZsq{}} \PY{o}{\PYZgt{}} \PY{l+s+s1}{\PYZsq{}}\PY{l+s+s1}{b}\PY{l+s+s1}{\PYZsq{}}
\end{Verbatim}


\begin{Verbatim}[commandchars=\\\{\}]
{\color{outcolor}Out[{\color{outcolor}190}]:} False
\end{Verbatim}
            
    \begin{Verbatim}[commandchars=\\\{\}]
{\color{incolor}In [{\color{incolor}191}]:} \PY{l+s+s1}{\PYZsq{}}\PY{l+s+s1}{b}\PY{l+s+s1}{\PYZsq{}} \PY{o}{\PYZgt{}} \PY{l+s+s1}{\PYZsq{}}\PY{l+s+s1}{a}\PY{l+s+s1}{\PYZsq{}}
\end{Verbatim}


\begin{Verbatim}[commandchars=\\\{\}]
{\color{outcolor}Out[{\color{outcolor}191}]:} True
\end{Verbatim}
            
    \begin{Shaded}
\begin{Highlighting}[]
\CommentTok{'''what do we want our program to do...  }
\CommentTok{Perhaps we could use a while loop?  }
\CommentTok{Let's think about the outcome.  }
\CommentTok{'azcbobobegghakl' is the example string.  }
\CommentTok{We'd go through this string and compare it to alphabetical order and collect the following substrings:  }
\CommentTok{'az' 'z' 'c' 'bo' 'o' 'bo' 'o' 'beggh' 'eggh' 'ggh' 'gh' 'h' 'akl' 'kl' 'l'}
\CommentTok{Then return 'beggh' because it is the largest.}
\CommentTok{Second example string: 'abcbcd'}
\CommentTok{'abc' 'bc' 'c' 'bcd' 'cd' 'd'}
\CommentTok{'abc' and 'bcd' are tied, but 'abc' is first so that is what is printed.}
\CommentTok{Whenever the next element of the string is "less" than the other, we want it to terminate. We don't even need to look at the alphabet.'''}
\end{Highlighting}
\end{Shaded}


    % Add a bibliography block to the postdoc
    
    
    
    \end{document}
